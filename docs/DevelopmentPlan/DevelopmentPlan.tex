\documentclass{article}

\usepackage{booktabs}
\usepackage{tabularx}

\title{Development Plan\\\progname}

\author{\authname}

\date{}

\input{../Comments}
%% Common Parts

\newcommand{\progname}{Natural Language Processing for Mental Health Risk Prediction} % PUT YOUR PROGRAM NAME HERE
\newcommand{\authname}{Team 13, The Cognitive Care Crew
\\ Jessica Dawson
\\ Michael Breau
\\ Matthew Curtis
\\ Benjamin Chinnery
\\ Yaruo Tian} % AUTHOR NAMES                  

\usepackage{hyperref}
    \hypersetup{colorlinks=true, linkcolor=blue, citecolor=blue, filecolor=blue,
                urlcolor=blue, unicode=false}
    \urlstyle{same}
                                


\begin{document}

\maketitle

\begin{table}[hp]
\caption{Revision History} \label{TblRevisionHistory}
\begin{tabularx}{\textwidth}{llX}
\toprule
\textbf{Date} & \textbf{Developer(s)} & \textbf{Change}\\
\midrule
Date1 & Name(s) & Description of changes\\
Date2 & Name(s) & Description of changes\\
... & ... & ...\\
\bottomrule
\end{tabularx}
\end{table}

\wss{Put your introductory blurb here.}

\section{Team Meeting Plan}

\section{Team Communication Plan}

\section{Team Member Roles}

\section{Workflow Plan}

\begin{itemize}
	\item How will you be using git, including branches, pull request, etc.?
	\item How will you be managing issues, including template issues, issue
	classificaiton, etc.?
\end{itemize}

\section{Proof of Concept Demonstration Plan}

What is the main risk, or risks, for the success of your project?  What will you
demonstrate during your proof of concept demonstration to convince yourself that
you will be able to overcome this risk?

\section{Technology}

\begin{table}[hp]
	\caption{Technology that will be used} \label{TblTechnology}
	\begin{tabularx}{\textwidth}{llX}
	\toprule
	\textbf{Technology} & \textbf{Use}\\
	\midrule
	GitHub & will be used as version control for both the docs and code repositories\\
	Python & will be used as the coding language for the majority of the project\\
	Pytest & will be used to validate code\\
	Pylint & will be used to ensure code quality\\
	GitHub Actions & will be used for continuous integration with Pytest and Pylint\\
	Stanza & Python library used for natural language processing\\
	\bottomrule
	\end{tabularx}
\end{table}

GitHub Actions will be used to run Pytest modules as well as check the code with Pylint when code is commited to the repository.
This will ensure the quality of code pushed to the repository.

\section{Coding Standard}

The Python code in this project will follow the PEP 8 coding standards.
This will be ensured by Pylint through extensions on the developers IDEs
as well as through the GitHub actions check when new code is pushed to the repository.

\section{Project Scheduling}

The project will largely follow the schedule outlined by the capstone course outline.
The schedule will be organized through the use of the kanban board on the team's GitHub
repository to ensure the team is on track. Team meetings will be scheduled through GitHub issues
which contain information regarding the details of meeting as well as being used to
keep track of attendance.


\end{document}