\documentclass[12pt, titlepage]{article}

\usepackage{booktabs}
\usepackage{tabularx}
\usepackage{hyperref}
\hypersetup{
    colorlinks,
    citecolor=blue,
    filecolor=black,
    linkcolor=red,
    urlcolor=blue
}
\usepackage[round]{natbib}

%% Comments

\usepackage{color}

\newif\ifcomments\commentstrue %displays comments
%\newif\ifcomments\commentsfalse %so that comments do not display

\ifcomments
\newcommand{\authornote}[3]{\textcolor{#1}{[#3 ---#2]}}
\newcommand{\todo}[1]{\textcolor{red}{[TODO: #1]}}
\else
\newcommand{\authornote}[3]{}
\newcommand{\todo}[1]{}
\fi

\newcommand{\wss}[1]{\authornote{blue}{SS}{#1}} 
\newcommand{\plt}[1]{\authornote{magenta}{TPLT}{#1}} %For explanation of the template
\newcommand{\an}[1]{\authornote{cyan}{Author}{#1}}

%% Common Parts

\newcommand{\progname}{Natural Language Processing for Mental Health Risk Prediction} % PUT YOUR PROGRAM NAME HERE
\newcommand{\authname}{Team 13, The Cognitive Care Crew
\\ Jessica Dawson
\\ Michael Breau
\\ Matthew Curtis
\\ Benjamin Chinnery
\\ Yaruo Tian} % AUTHOR NAMES                  

\usepackage{hyperref}
    \hypersetup{colorlinks=true, linkcolor=blue, citecolor=blue, filecolor=blue,
                urlcolor=blue, unicode=false}
    \urlstyle{same}
                                


\begin{document}

\title{Project Title: System Verification and Validation Plan for \progname{}} 
\author{\authname}
\date{\today}
	
\maketitle

\pagenumbering{roman}

\section*{Revision History}

\begin{tabularx}{\textwidth}{p{3cm}p{2cm}X}
\toprule {\bf Date} & {\bf Version} & {\bf Notes}\\
\midrule
Date 1 & 1.0 & Notes\\
Date 2 & 1.1 & Notes\\
\bottomrule
\end{tabularx}

~\\
\wss{The intention of the VnV plan is to increase confidence in the software.
However, this does not mean listing every verification and validation technique
that has ever been devised.  The VnV plan should also be a \textbf{feasible}
plan. Execution of the plan should be possible with the time and team available.
If the full plan cannot be completed during the time available, it can either be
modified to ``fake it'', or a better solution is to add a section describing
what work has been completed and what work is still planned for the future.}

\wss{The VnV plan is typically started after the requirements stage, but before
the design stage.  This means that the sections related to unit testing cannot
initially be completed.  The sections will be filled in after the design stage
is complete.  the final version of the VnV plan should have all sections filled
in.}

\newpage

\tableofcontents

\listoftables
\wss{Remove this section if it isn't needed}

\listoffigures
\wss{Remove this section if it isn't needed}

\newpage

\section{Symbols, Abbreviations, and Acronyms}

\renewcommand{\arraystretch}{1.2}
\begin{tabular}{l l} 
  \toprule		
  \textbf{symbol} & \textbf{description}\\
  \midrule 
  T & Test\\
  \bottomrule
\end{tabular}\\

\wss{symbols, abbreviations, or acronyms --- you can simply reference the SRS
  \citep{SRS} tables, if appropriate}

\wss{Remove this section if it isn't needed}

\newpage

\pagenumbering{arabic}

This document ... \wss{provide an introductory blurb and roadmap of the
  Verification and Validation plan}

\section{General Information}

\subsection{Summary}

\wss{Say what software is being tested.  Give its name and a brief overview of
  its general functions.}

\subsection{Objectives}

\wss{State what is intended to be accomplished.  The objective will be around
  the qualities that are most important for your project.  You might have
  something like: ``build confidence in the software correctness,''
  ``demonstrate adequate usability.'' etc.  You won't list all of the qualities,
  just those that are most important.}

\wss{You should also list the objectives that are out of scope.  You don't have 
the resources to do everything, so what will you be leaving out.  For instance, 
if you are not going to verify the quality of usability, state this.  It is also 
worthwhile to justify why the objectives are left out.}

\wss{The objectives are important because they highlight that you are aware of 
limitations in your resources for verification and validation.  You can't do everything, 
so what are you going to prioritize?  As an example, if your system depends on an 
external library, you can explicitly state that you will assume that external library 
has already been verified by its implementation team.}

\subsection{Relevant Documentation}

\wss{Reference relevant documentation.  This will definitely include your SRS
  and your other project documents (design documents, like MG, MIS, etc).  You
  can include these even before they are written, since by the time the project
  is done, they will be written.}

\citet{SRS}

\wss{Don't just list the other documents.  You should explain why they are relevant and 
how they relate to your VnV efforts.}

\section{Plan}

\wss{Introduce this section.   You can provide a roadmap of the sections to
  come.}

  \subsection{Verification and Validation Team}

  \wss{Your teammates.  Maybe your supervisor.
    You should do more than list names.  You should say what each person's role is
    for the project's verification.  A table is a good way to summarize this information.}\\
  
  The verification and validation team will consist of our core team/group members (Matthew, Jessica, Ben, Yuiro and Michael) along with our professor, our TA, Marie-Jean and Diego from our Montreal team, and Professor Mosser.  Our core team will be responsible for creating team suites that ensure correctness in our solution and that will catch possible bugs and issues that may arise. The team will be responsible for creating suitable edge cases to evaluate the correctness of our work along with general automated test suites that will be automatically deployed when new code is pulled.\\
  
  The core team will be responsible for creating all test suites, along with executing them and documenting the results. We will also be responsible for making any changes that are required after testing our code. All core team members will have a hand in all sections of testing but different team members will have different focused responsibilities. Firstly, all core team members will be responsible for documenting the results of the automated test suites when their code enters the repository through a pull request. More specifically, Yuiro and Micheals main responsibility will be creating a set of tests including edge cases for our NLP model that will ensure that our model functions as expected for a wide variety of input data. They will be required to create test suites along with automated test suites that will be run periodically when pull requests happen. Jessica, Ben and Matthew on the other hand will have the primary responsibility for training the data vs a training data set in order to determine the results and accuracy of the model. They will also have the responsibility to ensure code structure in the test suites that are created and organize the suites while Yuiro and Michaels main role is coming up with and creating the tests. \\
  
  The team will meet with Professor Mosser and their TA as well periodically throughout each checkpoint of design in order to discuss the requirements for our profile and documentations and to solidify what is expected of us. This is an opportunity for us to ask any questions we may have. We also will meet with Marie-Jean and Diego from the Montreal team periodically to help guide us on requirements and what validation means to them. They will help guide us to what the important things are to focus on within our project. Team 8 will also provide us feedback throughout every milestone. Lastly, the actual competition itself will be our final validation step when it steps our model against a new set of data and reports how our model performed. \\
  
  \subsection{SRS Verification Plan}
  
  \wss{List any approaches you intend to use for SRS verification.  This may include
    ad hoc feedback from reviewers, like your classmates, or you may plan for 
    something more rigorous/systematic.}
  
  \wss{Maybe create an SRS checklist?}\\
  
  All of the members of our core team will be taking part in the SRS Verification Plan along with Group 8, Professor Mosser and our TA. We will verify the contents of our SRS by comparing the requirements outlined by the eRisk competition and our team in Montreal with the requirements stated in our SRS. This will ensure that we have covered everything from our SRS and we can see if there is anything new we must add. Since our project is unique in the fact that we are submitting to a competition with a rigid structure of what we will be submitting, we will also compare the rubric for the SRS document with what we have done in our original SRS document to ensure that we hit all the needed checkpoints. On top of this, we will verify our SRS document by going over the feedback given to us on our SRS revision 0 from our TA along with the feedback we received from group 8 regarding our SRS revision 0. Lastly, when we are verifying our SRS document we will talk with Professor Mosser and ask him any questions that arise from us during the reviewing and verification process. 
  *Create an SRS checklist*
  
  
  
  
  \subsection{Design Verification Plan}
  
  \wss{Plans for design verification}
  
  \wss{The review will include reviews by your classmates}
  
  \wss{Create a checklists?}\\
  
  Our plan for our design verification will be to go through our SRS documents and make sure that each of our outputs and inputs that we planned on having are accounted for and included in our design planning and document. We will also look at reviews given to us from Group 8 on our SRS document to consider any changes we may want to add or things to consider for our design. We will go through the same process for feedback received from our TA on our SRS report. We will also utilize the guidance of Marie-Jean and Diego in order to verify that we are on the right path with our design. They can help guide us in our design and verify that it checks all the boxes and functionalities that are required for the competition. \\
  
  Below I will go over some design verification questions that will be asked regarding the functions of each of the major components in our design.\\
  
  \noindent \textbf{Text Pre-Processing Component:}
  \begin{itemize}
  \item Does the system tokenize text by dividing it into individual units of words and or sub-words?
  \item Does the system convert generated tokens into lowercase to preserve consistency?
  \item Does the system remove stop word tokens (common words that do not commonly have an effect on the meaning)?
  \item Does the system reduce tokens to their root forms (Ex: “moving” to “move”)?
  \item Is the output from this component usable by the Vectorization Component?\\
  \end{itemize}
  
  
  
  \noindent \textbf{Vectorization Component:}
  \begin{itemize}
  \item Does the system convert tokens into numerical vectors in this stage?
  \item Are these numerical vectors usable by the Prediction Component of the model?\\
  \end{itemize}
  
  \noindent \textbf{Prediction Component:}
  \begin{itemize}
  \item Is there a created training model within the design?
  \item Is there a prediction algorithm that is implemented into the solution? What is the level of accuracy and what is acceptable levels of accuracy (this will be determined further down the road with our team in Montreal on what we deem acceptable)?
  \item Does the system make predictions using the vectorization tokens, the training model, as well as with the created prediction algorithm?\\
  \end{itemize}
  
  
  \noindent \textbf{Output Component:}
  \begin{itemize}
  \item Does the system produce the output of the predictions to the console?
  \item Is the output produced understandable by the project team?
  \item Is the output in a format that a healthcare worker would be able to read and interpret the results?
  \end{itemize}
  \\
  
  
  \subsection{Verification and Validation Plan Verification Plan}
  
  \wss{The verification and validation plan is an artifact that should also be
  verified.  Techniques for this include review and mutation testing.}
  
  \wss{The review will include reviews by your classmates}
  
  \wss{Create a checklists?}\\
  
  For our Verification and Validation Plan Verification Plan we will be making use of the reviews given to us for our VnV Plan from group 8 as well as from our TA. This will help us gain some insight on areas that may be missing or need more work. We will also be going through our SRS and ensuring we have created at least 1 test for each functional and nonfunctional requirement. It is important to ensure that we have sufficient tests for our requirements. We would also like to do some mutation testing in our code to determine what changes will affect the tests we have created and in what way. It will also show us that there is a possibility we are missing some tests that should be included if a mutation that we would expect to result in a change, results in no failed tests.
  
  Below is a checklist of the type of questions and topics we would like to cover during out Verification and Validation Plan Verification Plan.\\
  
  \begin{itemize}
  
  \item Is the brief summary of the software in the summary section enough to get an understanding of what the software is that is being tested and its general functions?
  \item Are the objectives for the VnV plan clearly outlined and is each objective covered at some point throughout the document?
  \item Is the planning section detailed with checklists for each section and steps that will be taken for each plan?
  \item Is there at least one test created for each requirement in the SRS document?
  \item Are all functions of each component being tested? This includes the Text Pre-Processing, Vectorization, Prediction and Output Components.
  \item Are there a wide variety of tests including edge cases?
  Do the non-functional requirements contain tangible quantifiable values?
  \item Is the traceability between test cases and requirements clearly shown in section 4.3?
  \end{itemize}
  
  
  \subsection{Implementation Verification Plan}
  
  \wss{You should at least point to the tests listed in this document and the unit
    testing plan.}
  
  \wss{In this section you would also give any details of any plans for static
    verification of the implementation.  Potential techniques include code
    walkthroughs, code inspection, static analyzers, etc.}
  
  \subsection{Automated Testing and Verification Tools}
  
  \wss{What tools are you using for automated testing.  Likely a unit testing
    framework and maybe a profiling tool, like ValGrind.  Other possible tools
    include a static analyzer, make, continuous integration tools, test coverage
    tools, etc.  Explain your plans for summarizing code coverage metrics.
    Linters are another important class of tools.  For the programming language
    you select, you should look at the available linters.  There may also be tools
    that verify that coding standards have been respected, like flake9 for
    Python.}
  
  \wss{If you have already done this in the development plan, you can point to
  that document.}
  
  \wss{The details of this section will likely evolve as you get closer to the
    implementation.}\\
  
  There will be three main tools we will be using for automated testing and verification. Firstly, we will be using Flake9 which is a fork of Flake8. This will help us in following proper coding standards and prevent problems arising such as syntax errors, typos, incorrect styling, bad formatting and more. This will also help save time for when we do peer code reviews as a team. We will also be creating testing suites using Pytest since it is a free open-sourced, simple and scalable Python-based Test Automation Framework. We will be using GitHub Actions in order to implement continuous integration and continuous delivery into our project. This will allow us to automatically build, test, and deploy our pipeline. This will create a pipeline that allows us to build and test every pull request when they happen and also when we deploy and merge a pull request onto another branch.
  
\subsection{Software Validation Plan}

\wss{If there is any external data that can be used for validation, you should
  point to it here.  If there are no plans for validation, you should state that
  here.}

\wss{You might want to use review sessions with the stakeholder to check that
the requirements document captures the right requirements.  Maybe task based
inspection?}

\wss{For those capstone teams with an external supervisor, the Rev 0 demo should 
be used as an opportunity to validate the requirements.  You should plan on 
demonstrating your project to your supervisor shortly after the scheduled Rev 0 demo.  
The feedback from your supervisor will be very useful for improving your project.}

\wss{For teams without an external supervisor, user testing can serve the same purpose 
as a Rev 0 demo for the supervisor.}

\wss{This section might reference back to the SRS verification section.}

\section{System Test Description}
	
\subsection{Tests for Functional Requirements}

\wss{Subsets of the tests may be in related, so this section is divided into
  different areas.  If there are no identifiable subsets for the tests, this
  level of document structure can be removed.}

\wss{Include a blurb here to explain why the subsections below
  cover the requirements.  References to the SRS would be good here.}

\subsubsection{Area of Testing1}

\wss{It would be nice to have a blurb here to explain why the subsections below
  cover the requirements.  References to the SRS would be good here.  If a section
  covers tests for input constraints, you should reference the data constraints
  table in the SRS.}
		
\paragraph{Title for Test}

\begin{enumerate}

\item{test-id1\\}

Control: Manual versus Automatic
					
Initial State: 
					
Input: 
					
Output: \wss{The expected result for the given inputs}

Test Case Derivation: \wss{Justify the expected value given in the Output field}
					
How test will be performed: 
					
\item{test-id2\\}

Control: Manual versus Automatic
					
Initial State: 
					
Input: 
					
Output: \wss{The expected result for the given inputs}

Test Case Derivation: \wss{Justify the expected value given in the Output field}

How test will be performed: 

\end{enumerate}

\subsubsection{Area of Testing2}

...

\subsection{Tests for Nonfunctional Requirements}

\wss{The nonfunctional requirements for accuracy will likely just reference the
  appropriate functional tests from above.  The test cases should mention
  reporting the relative error for these tests.  Not all projects will
  necessarily have nonfunctional requirements related to accuracy}

\wss{Tests related to usability could include conducting a usability test and
  survey.  The survey will be in the Appendix.}

\wss{Static tests, review, inspections, and walkthroughs, will not follow the
format for the tests given below.}

\subsubsection{Area of Testing1}
		
\paragraph{Title for Test}

\begin{enumerate}

\item{test-id1\\}

Type: Functional, Dynamic, Manual, Static etc.
					
Initial State: 
					
Input/Condition: 
					
Output/Result: 
					
How test will be performed: 
					
\item{test-id2\\}

Type: Functional, Dynamic, Manual, Static etc.
					
Initial State: 
					
Input: 
					
Output: 
					
How test will be performed: 

\end{enumerate}

\subsubsection{Area of Testing2}

...

\subsection{Traceability Between Test Cases and Requirements}

\wss{Provide a table that shows which test cases are supporting which
  requirements.}

\section{Unit Test Description}

\wss{This section should not be filled in until after the MIS (detailed design
  document) has been completed.}

\wss{Reference your MIS (detailed design document) and explain your overall
philosophy for test case selection.}  

\wss{To save space and time, it may be an option to provide less detail in this section.  
For the unit tests you can potentially layout your testing strategy here.  That is, you 
can explain how tests will be selected for each module.  For instance, your test building 
approach could be test cases for each access program, including one test for normal behaviour 
and as many tests as needed for edge cases.  Rather than create the details of the input 
and output here, you could point to the unit testing code.  For this to work, you code 
needs to be well-documented, with meaningful names for all of the tests.}

\subsection{Unit Testing Scope}

\wss{What modules are outside of the scope.  If there are modules that are
  developed by someone else, then you would say here if you aren't planning on
  verifying them.  There may also be modules that are part of your software, but
  have a lower priority for verification than others.  If this is the case,
  explain your rationale for the ranking of module importance.}

\subsection{Tests for Functional Requirements}

\wss{Most of the verification will be through automated unit testing.  If
  appropriate specific modules can be verified by a non-testing based
  technique.  That can also be documented in this section.}

\subsubsection{Module 1}

\wss{Include a blurb here to explain why the subsections below cover the module.
  References to the MIS would be good.  You will want tests from a black box
  perspective and from a white box perspective.  Explain to the reader how the
  tests were selected.}

\begin{enumerate}

\item{test-id1\\}

Type: \wss{Functional, Dynamic, Manual, Automatic, Static etc. Most will
  be automatic}
					
Initial State: 
					
Input: 
					
Output: \wss{The expected result for the given inputs}

Test Case Derivation: \wss{Justify the expected value given in the Output field}

How test will be performed: 
					
\item{test-id2\\}

Type: \wss{Functional, Dynamic, Manual, Automatic, Static etc. Most will
  be automatic}
					
Initial State: 
					
Input: 
					
Output: \wss{The expected result for the given inputs}

Test Case Derivation: \wss{Justify the expected value given in the Output field}

How test will be performed: 

\item{...\\}
    
\end{enumerate}

\subsubsection{Module 2}

...

\subsection{Tests for Nonfunctional Requirements}

\wss{If there is a module that needs to be independently assessed for
  performance, those test cases can go here.  In some projects, planning for
  nonfunctional tests of units will not be that relevant.}

\wss{These tests may involve collecting performance data from previously
  mentioned functional tests.}

\subsubsection{Module ?}
		
\begin{enumerate}

\item{test-id1\\}

Type: \wss{Functional, Dynamic, Manual, Automatic, Static etc. Most will
  be automatic}
					
Initial State: 
					
Input/Condition: 
					
Output/Result: 
					
How test will be performed: 
					
\item{test-id2\\}

Type: Functional, Dynamic, Manual, Static etc.
					
Initial State: 
					
Input: 
					
Output: 
					
How test will be performed: 

\end{enumerate}

\subsubsection{Module ?}

...

\subsection{Traceability Between Test Cases and Modules}

\wss{Provide evidence that all of the modules have been considered.}
				
\bibliographystyle{plainnat}

\bibliography{../../refs/References}

\newpage

\section{Appendix}

This is where you can place additional information.

\subsection{Symbolic Parameters}

The definition of the test cases will call for SYMBOLIC\_CONSTANTS.
Their values are defined in this section for easy maintenance.

\subsection{Usability Survey Questions?}

\wss{This is a section that would be appropriate for some projects.}

\newpage{}
\section*{Appendix --- Reflection}

The information in this section will be used to evaluate the team members on the
graduate attribute of Lifelong Learning.  Please answer the following questions:

\newpage{}
\section*{Appendix --- Reflection}

\wss{This section is not required for CAS 741}

The information in this section will be used to evaluate the team members on the
graduate attribute of Lifelong Learning.  Please answer the following questions:

\begin{enumerate}
  \item What knowledge and skills will the team collectively need to acquire to
  successfully complete the verification and validation of your project?
  Examples of possible knowledge and skills include dynamic testing knowledge,
  static testing knowledge, specific tool usage etc.  You should look to
  identify at least one item for each team member.
  \item For each of the knowledge areas and skills identified in the previous
  question, what are at least two approaches to acquiring the knowledge or
  mastering the skill?  Of the identified approaches, which will each team
  member pursue, and why did they make this choice?
\end{enumerate}

\end{document}