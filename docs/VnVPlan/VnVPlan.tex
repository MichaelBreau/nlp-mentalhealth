\documentclass[12pt, titlepage]{article}

\usepackage{booktabs}
\usepackage{tabularx}
\usepackage{hyperref}
\hypersetup{
    colorlinks,
    citecolor=blue,
    filecolor=black,
    linkcolor=red,
    urlcolor=blue
}
\usepackage[round]{natbib}

%% Comments

\usepackage{color}

\newif\ifcomments\commentstrue %displays comments
%\newif\ifcomments\commentsfalse %so that comments do not display

\ifcomments
\newcommand{\authornote}[3]{\textcolor{#1}{[#3 ---#2]}}
\newcommand{\todo}[1]{\textcolor{red}{[TODO: #1]}}
\else
\newcommand{\authornote}[3]{}
\newcommand{\todo}[1]{}
\fi

\newcommand{\wss}[1]{\authornote{blue}{SS}{#1}} 
\newcommand{\plt}[1]{\authornote{magenta}{TPLT}{#1}} %For explanation of the template
\newcommand{\an}[1]{\authornote{cyan}{Author}{#1}}

%% Common Parts

\newcommand{\progname}{Natural Language Processing for Mental Health Risk Prediction} % PUT YOUR PROGRAM NAME HERE
\newcommand{\authname}{Team 13, The Cognitive Care Crew
\\ Jessica Dawson
\\ Michael Breau
\\ Matthew Curtis
\\ Benjamin Chinnery
\\ Yaruo Tian} % AUTHOR NAMES                  

\usepackage{hyperref}
    \hypersetup{colorlinks=true, linkcolor=blue, citecolor=blue, filecolor=blue,
                urlcolor=blue, unicode=false}
    \urlstyle{same}
                                


\begin{document}

\title{Project Title: System Verification and Validation Plan for \progname{}} 
\author{\authname}
\date{\today}
	
\maketitle

\pagenumbering{roman}

\section*{Revision History}

\begin{tabularx}{\textwidth}{p{3cm}p{2cm}X}
\toprule {\bf Date} & {\bf Version} & {\bf Notes}\\
\midrule
Date 1 & 1.0 & Notes\\
Date 2 & 1.1 & Notes\\
\bottomrule
\end{tabularx}

~\\
\wss{The intention of the VnV plan is to increase confidence in the software.
However, this does not mean listing every verification and validation technique
that has ever been devised.  The VnV plan should also be a \textbf{feasible}
plan. Execution of the plan should be possible with the time and team available.
If the full plan cannot be completed during the time available, it can either be
modified to ``fake it'', or a better solution is to add a section describing
what work has been completed and what work is still planned for the future.}

\wss{The VnV plan is typically started after the requirements stage, but before
the design stage.  This means that the sections related to unit testing cannot
initially be completed.  The sections will be filled in after the design stage
is complete.  the final version of the VnV plan should have all sections filled
in.}

\newpage

\tableofcontents

\listoftables
\wss{Remove this section if it isn't needed}

\listoffigures
\wss{Remove this section if it isn't needed}

\newpage

\section{Symbols, Abbreviations, and Acronyms}

\renewcommand{\arraystretch}{1.2}
\begin{tabular}{l l} 
  \toprule		
  \textbf{symbol} & \textbf{description}\\
  \midrule 
  T & Test\\
  \bottomrule
\end{tabular}\\

\wss{symbols, abbreviations, or acronyms --- you can simply reference the SRS
  \citep{SRS} tables, if appropriate}

\wss{Remove this section if it isn't needed}

\newpage

\pagenumbering{arabic}

This document ... \wss{provide an introductory blurb and roadmap of the
  Verification and Validation plan}

\section{General Information}

\subsection{Summary}

\wss{Say what software is being tested.  Give its name and a brief overview of
  its general functions.}

\subsection{Objectives}

\wss{State what is intended to be accomplished.  The objective will be around
  the qualities that are most important for your project.  You might have
  something like: ``build confidence in the software correctness,''
  ``demonstrate adequate usability.'' etc.  You won't list all of the qualities,
  just those that are most important.}

\wss{You should also list the objectives that are out of scope.  You don't have 
the resources to do everything, so what will you be leaving out.  For instance, 
if you are not going to verify the quality of usability, state this.  It is also 
worthwhile to justify why the objectives are left out.}

\wss{The objectives are important because they highlight that you are aware of 
limitations in your resources for verification and validation.  You can't do everything, 
so what are you going to prioritize?  As an example, if your system depends on an 
external library, you can explicitly state that you will assume that external library 
has already been verified by its implementation team.}

\subsection{Relevant Documentation}

\wss{Reference relevant documentation.  This will definitely include your SRS
  and your other project documents (design documents, like MG, MIS, etc).  You
  can include these even before they are written, since by the time the project
  is done, they will be written.}

\citet{SRS}

\wss{Don't just list the other documents.  You should explain why they are relevant and 
how they relate to your VnV efforts.}

\section{Plan}

This section will outline the plan for verification and validation of a number of different components within this project  (\ref{Verification and Validation Team}). This section will first give an overview of the members that are considered part of the verification and validation team and what their roles and involvement will consist of. Next there will be subsections regarding verification of the project's \href{https://github.com/MichaelBreau/nlp-mentalhealth/blob/main/docs/SRS/index.pdf}{SRS} (\ref{SRS Verification Plan}) along with the design plan (\ref{Design Verification Plan}) and the verification and validation plan itself (\ref{Verification and Validation Plan Verification Plan}). This section will then outline the verification plan that has been created for the implementation of the project and model (\ref{Implementation Verification Plan}). Automated Testing and verification tools that will be used throughout this project is then the next subsection that is outlined in \ref{Automated Testing and Verification Tools} followed lastly by the validation plan for our software (\ref{Software Validation Plan}).
  
  \subsection{Verification and Validation Team} \label{Verification and Validation Team}

  
  The verification and validation team will consist of our core team members (Matthew, Jessica, Ben, Yuiro and Michael) along with our professor, our TA, Marie-Jean and Diego from our Montreal team, and Professor Mosser.  Our core team will be responsible for creating test suites that ensure correctness in our solution and that will catch possible bugs and issues that may arise. The team will be responsible for creating suitable edge cases to evaluate the correctness of our work along with general automated test suites that will be automatically deployed when new code is pulled.\\
  
  The core team will be responsible for creating all test suites, along with executing them and documenting the results. We will also be responsible for making any changes that are required after testing our code. All core team members will have a hand in all sections of testing but different team members will have different focused responsibilities. Firstly, all core team members will be responsible for documenting the results of the automated test suites when their code enters the repository through a pull request. More specifically, Yuiro and Micheals main responsibility will be creating a set of tests including edge cases for our NLP model that will ensure that our model functions as expected for a wide variety of input data. They will be required to create test suites along with automated test suites that will be run periodically when pull requests happen. Jessica, Ben and Matthew on the other hand will have the primary responsibility for training the data vs a training data set in order to determine the accuracy of the model. They will also have the responsibility to ensure code structure in the test suites that are created along with organizing the suites while Yuiro and Michaels main role is coming up with and creating the tests.\\ 
  
  The team will meet with Professor Mosser and their TA as well periodically throughout each milestone in order to discuss the requirements for our project regarding the current milestone and to help solidify what is expected of us. This is an opportunity for us to ask any questions we may have as well. We also will meet with Marie-Jean and Diego from the Montreal team periodically to help guide us with regard to requirements of our project and the eRisk competition along with what validation means to them. They will help guide us to what the important things are to focus on within our project. Team 8 will also provide us feedback for every milestone. Lastly, the actual competition itself will be our final validation step when it tests our model against a new set of data and reports how our model performs.\\

  
  \subsection{SRS Verification Plan} \label{SRS Verification Plan}
  
  All of the members of our core team will be taking part in the SRS Verification Plan along with Group 8, Professor Mosser and our TA. We will verify the contents of our \href{https://github.com/MichaelBreau/nlp-mentalhealth/blob/main/docs/SRS/index.pdf}{SRS} by comparing the requirements outlined by the eRisk competition along with our team in Montreal with the requirements stated in our SRS document. This will ensure that we have covered everything from our SRS document and we can see if there is anything new we must add. Most of our SRS verification plan will involve inspections from us but also from the team in Montreal along with peer reviews and ad hoc feedback from group 8.\\
  
  Since our project is unique in the fact that we are submitting to a competition with a rigid output and guideline structure, we will also have to compare the rubric for the SRS document with what we have done in our original SRS document to ensure that we have covered all the required checkpoints. On top of this, we will verify our SRS document by going over the feedback given to us on our SRS revision 0 from our TA along with the feedback we received from group 8 regarding our SRS revision 0. Lastly, when we are verifying our SRS document we will talk with Professor Mosser and ask him any questions that arise during the reviewing and verification process within the group. \\

  These are some of the major areas we would like to cover in our SRS verification plan to ensure we have a test for each of these components:\\

  \begin{itemize}
  \item Is a functional overview of the system provided in the SRS?
  \item Are high-level usage scenarios included?
  \item Have we specified the software environment and all its elements?
  \item Are the limitations and exclusions accurate and all encompassing of the true limitations and exclusions of our project?
  \item Are the inputs and outputs of the system specified?
  \item Is there any unnecessary overlap in our requirements?
  \item Are there any other components outside of Text Pre-Processing, Vectorization, Prediction and Output? For those 4 sections, are there any additional functional or non-functional requirements that are missing?
  \item For every function that is outlined, are the inputs specified sufficient enough to perform the required functionality?
  \item Do any requirements conflict with each other or does each requirement avoid conflicts with other requirements?
  \item Are all detailed usage scenarios covered in our document and are completed in adequate detail or are there others we could add?
  \item Does each component and requirement have a priority?
  \item Are there any chances in our prioritization chart we should add due to changes in our code since the creation of our original SRS (revision 0)?
  \item Are the requirements clear enough that it would be possible to give to an independent team for implementation and they would be able to understand the requirements?
  \item Is every requirement testable?
  \item Is the verification and acceptance criteria specific enough? If not, how can you add quantifiable metrics to each point?
  \item Have we defined every definition, abbreviation and acronym in our Glossary?
  \item Are all imposed technical choices listed in the document?
  \item Are the risks that may be present in this project along with their mitigation analysis included in this document?
  \item Is the “Requirements Process and Reports” accurate to what is expected by each of the teams listed in the “Requirements Process and Reports” section?
  \item Does the Context and Overall Objectives section paint an accurate enough description of the project's context and objectives?\\
  \end{itemize}
  
  
  \subsection{Design Verification Plan} \label{Design Verification Plan}
  
  Our plan for our design verification will be to go through our SRS document and make sure that each of our outputs and inputs that we planned on having are accounted for and included in our design planning and document. We will also look at reviews given to us by Group 8 on our SRS document to consider any changes we may want to add or things to consider for our design. We will go through the same process for feedback received from our TA on our SRS report. We will also utilize the guidance of Marie-Jean and Diego in order to verify that we are on the right path with our design. They can help guide us in our design and verify that it checks all the boxes and functionalities that are required for the competition. \\

  Below I will go over some design verification questions that will be asked regarding the functions of each of the major components in our design.\\
  
  \noindent \textbf{Text Pre-Processing Component:}
  \begin{itemize}
  \item Does the system tokenize text by dividing it into individual units of words and or sub-words?
  \item Does the system convert generated tokens into lowercase to preserve consistency?
  \item Does the system remove stop word tokens (common words that do not commonly have an effect on the meaning)?
  \item Does the system reduce tokens to their root forms (Ex: “moving” to “move”)?
  \item Is the output from this component usable by the Vectorization Component?\\
  \end{itemize}
  
  
  
  \noindent \textbf{Vectorization Component:}
  \begin{itemize}
  \item Does the system convert tokens into numerical vectors in this stage?
  \item Are these numerical vectors usable by the Prediction Component of the model?\\
  \end{itemize}
  
  \noindent \textbf{Prediction Component:}
  \begin{itemize}
  \item Is there a created training model within the design?
  \item What percentage of the given data is used for training vs testing?
  \item Is there a prediction algorithm that is implemented into the solution? What is the level of accuracy and what is acceptable levels of accuracy (this will be determined further down the road with our team in Montreal on what we deem acceptable)?
  \item Does the system make predictions using the vectorization tokens, the training model, as well as with the created prediction algorithm?\\
  \end{itemize}
  
  
  \noindent \textbf{Output Component:}
  \begin{itemize}
  \item Does the system output the predictions to the console?
  \item Is the output produced understandable by the project team?
  \item Is the output in a format that a healthcare worker would be able to read and interpret the results?
  \end{itemize}
  \\
  
  
  \subsection{Verification and Validation Plan Verification Plan}\label{Verification and Validation Plan Verification Plan}
  
  For our Verification and Validation Plan Verification Plan we will be making use of the reviews given to us for our VnV Plan from group 8 as well as from our TA. This will help us gain some insight on areas that may be missing or need more work within the document. We will also be going through our \href{https://github.com/MichaelBreau/nlp-mentalhealth/blob/main/docs/SRS/index.pdf}{SRS} document and ensuring we have created at least 1 test for each functional and nonfunctional requirement. It is important to ensure that we have sufficient tests for our requirements. We would also like to do some mutation testing in our code to determine what changes will affect the tests we have created and in what way. It will also show us that there is a possibility we are missing some tests that should be included if a mutation that we would expect to result in a change or failure, results in no failed tests.\\ 

  Below is a checklist of the type of questions and topics our team would like to cover as part of our Verification and Validation Plan Verification Plan.\\
  
  \begin{itemize}
  
  \item Is the brief summary of the software in the summary section enough to get an understanding of the software that is being tested and its general functions?
  \item Are the objectives for the VnV plan clearly outlined and is each objective covered at some point throughout the document?
  \item Is the planning section detailed with checklists for each section and steps that will be taken for each plan?
  \item Is there at least one test created for each requirement in the SRS document?
  \item Are all functions of each component being tested? This includes the Text Pre-Processing, Vectorization, Prediction and Output Components.
  \item Are there a wide variety of tests including edge cases?
  Do the non-functional requirements contain tangible quantifiable values?
  \item Is the traceability between test cases and requirements clearly shown in section 4.3?
  \end{itemize}
  
  
  \subsection{Implementation Verification Plan} \label{Implementation Verification Plan}
  
  The large majority of the implementation for this project will revolve around achieving an acceptable level of accuracy in our system’s predictions. This document goes more in depth about how we will do this in our implementation in section 4.1.1 but the basis of this process to ensure acceptable model accuracy levels is done through train/test splits, training models on the test sets, predicting the values of test data with these models, and then comparing these predicted values with the known expected values. All of the sections just listed are key areas of the implementation and will need to be tested in order to verify our implementation. There are 4 tests that will be run, typically in tandem with each other in one automatic test suite as outlined in 4.1.1. The largest area will be involving training and testing the data. Data will be provided by eRisk and the data will be used 3 times using different splits for each copy regarding what data is used for training vs testing. These predictions on the testing data will then be evaluated on a relevant set of metrics in order to test the program's accuracy and implementation. This set of metrics has yet to be determined due to the fact that the eRisk tasks have not been released yet for this year so we are unclear what metrics would be appropriate to evaluate our model at the current time. We will also be running a series of unit tests on our program in order to test the implementation of each part and function of the code. These tests will be recorded in the unit testing plan.\\
    
  As new code is added to our repository and as we work on our model, we will be using implementation verification techniques like code walk-throughs and code inspections periodically. Every time a new pull request is made, a fellow teammate will be required to perform a code inspection for that pull request and give feedback. If there is something unclear, the two teammates will meet and the reviewer will receive a code walk-through from the reviewee. We will also perform code walk-throughs for new changes made to the model when we meet as a team. Lastly we plan to have continuous integration that will allow us to run automated testing for every pull request that gets created as well as utilizing Fake9 as mentioned below to ensure our implementation follows proper coding standards.
  
  
  \subsection{Automated Testing and Verification Tools} \label{Automated Testing and Verification Tools}

  There will be three main tools the team will be using for automated testing and verification. Firstly, we will be using Flake9 which is a fork of Flake8. This will help us in following proper coding standards and prevent problems arising regarding topics such as syntax errors, typos, incorrect styling, bad formatting and more. This will also help save time for when we do peer code reviews as a team. We will also be creating test suites using Pytest since it is a free open-sourced, simple and scalable Python-based Test Automation Framework. We will be using GitHub Actions in order to implement continuous integration and continuous delivery into our project. This will allow us to automatically build, test, and deploy our pipeline. This will create a pipeline that allows us to build and test every pull request when they happen and also when we deploy and merge a pull request onto another branch.
  
  \subsection{Software Validation Plan} \label{Software Validation Plan}
  
  There are a multitude of methods and parts to our software validation plan. Firstly, our primary validation method will be the actual process of competing in the eRisk competition. During the competition they will take our model and test it with new data and observe the output. This will be the final and largest validation step for us that will help validate the correctness and precision of our design. We will also have a set of training data that we will use throughout the design process to train against that we can use to test how close our results are to the expected results, validating how accurate our design is. \\

  Another large aspect of the software validation plan will be our review sessions and meeting with Marie-Jean and Diego throughout the project. This will be a time for them to look at our software and run it themselves. A large checkpoint when they will have an opportunity to have our project demoed for them is around the Rev 0 demo. This will give them an opportunity to validate that our project passes all the requirements of the competition or if there is something else we need to add or change. This is an important time for us to take this feedback and use it in order to improve our project. This will also be a time for us to ask questions and for them to confirm the requirements are all being met in our design. These questions and requirements would largely surround the overall output of the design and the overall process (largely focusing on the training of our data and our prediction model we create and use). It would also include validation of each of the program's components individually and as a collective (components being the Text Pre-Processing, Vectorization, Prediction and Output components). \\

  Before that however, we will meet with professor Mosser to discuss if he thinks that all the requirements documented line up with and are accurate to the requirements outlined for the course. We will do this with Marie-Jean and Diego as well with respect to the requirements from the competition. We will also go over the SRS document one last time to ensure that every requirement is met. \\  
  
  Regarding our demo with Professor Mosser, Marie-Jean and Diego we will include the following questions:
  
  \begin{itemize}
  \item Is the output format what is expected by the competition?
  \item Are there any requirements not being met within the competition (question for Marie-Jean and Diego)?
  \item Are there any requirements that are not being met regarding the project rubric from McMaster (question for Professor Mosser)?
  \item Is our text pre-processing component sufficient in which we use tokenization, lowercasing, stopword removal, and stemming or Lemmatization? If not, what should we add to our text pre-processing component or what methods that we currently have employed are not needed?
  \item Is our current method of vectorization appropriate for the task we are completing in your opinion? If not, what would you change or what different approaches do you recommend we look into.
  \item Are there any areas of our code or processes we take that are not allowed in the competition or do not follow the competitions standards?
  \item What are your thoughts on our topic modeling approach? Do you have any suggestions?
  \item Are the metrics we use for determining accuracy appropriate in your opinion? If not, what form of metrics should we be using? What level of accuracy would you consider acceptable for this metric?
  \item In your opinion is the run time of our program to output the results acceptable? If not, what would be an acceptable time?\end{itemize}
  

\section{System Test Description}
	
\subsection{Tests for Functional Requirements}

\wss{Subsets of the tests may be in related, so this section is divided into
  different areas.  If there are no identifiable subsets for the tests, this
  level of document structure can be removed.}

\wss{Include a blurb here to explain why the subsections below
  cover the requirements.  References to the SRS would be good here.}

\subsubsection{Area of Testing1}

\wss{It would be nice to have a blurb here to explain why the subsections below
  cover the requirements.  References to the SRS would be good here.  If a section
  covers tests for input constraints, you should reference the data constraints
  table in the SRS.}
		
\paragraph{Title for Test}

\begin{enumerate}

\item{test-id1\\}

Control: Manual versus Automatic
					
Initial State: 
					
Input: 
					
Output: \wss{The expected result for the given inputs}

Test Case Derivation: \wss{Justify the expected value given in the Output field}
					
How test will be performed: 
					
\item{test-id2\\}

Control: Manual versus Automatic
					
Initial State: 
					
Input: 
					
Output: \wss{The expected result for the given inputs}

Test Case Derivation: \wss{Justify the expected value given in the Output field}

How test will be performed: 

\end{enumerate}

\subsubsection{Area of Testing2}

...

\subsection{Tests for Nonfunctional Requirements}

\wss{The nonfunctional requirements for accuracy will likely just reference the
  appropriate functional tests from above.  The test cases should mention
  reporting the relative error for these tests.  Not all projects will
  necessarily have nonfunctional requirements related to accuracy}

\wss{Tests related to usability could include conducting a usability test and
  survey.  The survey will be in the Appendix.}

\wss{Static tests, review, inspections, and walkthroughs, will not follow the
format for the tests given below.}

\subsubsection{Area of Testing1}
		
\paragraph{Title for Test}

\begin{enumerate}

\item{test-id1\\}

Type: Functional, Dynamic, Manual, Static etc.
					
Initial State: 
					
Input/Condition: 
					
Output/Result: 
					
How test will be performed: 
					
\item{test-id2\\}

Type: Functional, Dynamic, Manual, Static etc.
					
Initial State: 
					
Input: 
					
Output: 
					
How test will be performed: 

\end{enumerate}

\subsubsection{Area of Testing2}

...

\subsection{Traceability Between Test Cases and Requirements}

\wss{Provide a table that shows which test cases are supporting which
  requirements.}

\section{Unit Test Description}

\wss{This section should not be filled in until after the MIS (detailed design
  document) has been completed.}

\wss{Reference your MIS (detailed design document) and explain your overall
philosophy for test case selection.}  

\wss{To save space and time, it may be an option to provide less detail in this section.  
For the unit tests you can potentially layout your testing strategy here.  That is, you 
can explain how tests will be selected for each module.  For instance, your test building 
approach could be test cases for each access program, including one test for normal behaviour 
and as many tests as needed for edge cases.  Rather than create the details of the input 
and output here, you could point to the unit testing code.  For this to work, you code 
needs to be well-documented, with meaningful names for all of the tests.}

\subsection{Unit Testing Scope}

\wss{What modules are outside of the scope.  If there are modules that are
  developed by someone else, then you would say here if you aren't planning on
  verifying them.  There may also be modules that are part of your software, but
  have a lower priority for verification than others.  If this is the case,
  explain your rationale for the ranking of module importance.}

\subsection{Tests for Functional Requirements}

\wss{Most of the verification will be through automated unit testing.  If
  appropriate specific modules can be verified by a non-testing based
  technique.  That can also be documented in this section.}

\subsubsection{Module 1}

\wss{Include a blurb here to explain why the subsections below cover the module.
  References to the MIS would be good.  You will want tests from a black box
  perspective and from a white box perspective.  Explain to the reader how the
  tests were selected.}

\begin{enumerate}

\item{test-id1\\}

Type: \wss{Functional, Dynamic, Manual, Automatic, Static etc. Most will
  be automatic}
					
Initial State: 
					
Input: 
					
Output: \wss{The expected result for the given inputs}

Test Case Derivation: \wss{Justify the expected value given in the Output field}

How test will be performed: 
					
\item{test-id2\\}

Type: \wss{Functional, Dynamic, Manual, Automatic, Static etc. Most will
  be automatic}
					
Initial State: 
					
Input: 
					
Output: \wss{The expected result for the given inputs}

Test Case Derivation: \wss{Justify the expected value given in the Output field}

How test will be performed: 

\item{...\\}
    
\end{enumerate}

\subsubsection{Module 2}

...

\subsection{Tests for Nonfunctional Requirements}

\wss{If there is a module that needs to be independently assessed for
  performance, those test cases can go here.  In some projects, planning for
  nonfunctional tests of units will not be that relevant.}

\wss{These tests may involve collecting performance data from previously
  mentioned functional tests.}

\subsubsection{Module ?}
		
\begin{enumerate}

\item{test-id1\\}

Type: \wss{Functional, Dynamic, Manual, Automatic, Static etc. Most will
  be automatic}
					
Initial State: 
					
Input/Condition: 
					
Output/Result: 
					
How test will be performed: 
					
\item{test-id2\\}

Type: Functional, Dynamic, Manual, Static etc.
					
Initial State: 
					
Input: 
					
Output: 
					
How test will be performed: 

\end{enumerate}

\subsubsection{Module ?}

...

\subsection{Traceability Between Test Cases and Modules}

\wss{Provide evidence that all of the modules have been considered.}
				
\bibliographystyle{plainnat}

\bibliography{../../refs/References}

\newpage

\section{Appendix}

This is where you can place additional information.

\subsection{Symbolic Parameters}

The definition of the test cases will call for SYMBOLIC\_CONSTANTS.
Their values are defined in this section for easy maintenance.

\subsection{Usability Survey Questions?}

\wss{This is a section that would be appropriate for some projects.}

\newpage{}
\section*{Appendix --- Reflection}

The information in this section will be used to evaluate the team members on the
graduate attribute of Lifelong Learning.  Please answer the following questions:

\newpage{}
\section*{Appendix --- Reflection}

\wss{This section is not required for CAS 741}

The information in this section will be used to evaluate the team members on the
graduate attribute of Lifelong Learning.  Please answer the following questions:

\begin{enumerate}
  \item What knowledge and skills will the team collectively need to acquire to
  successfully complete the verification and validation of your project?
  Examples of possible knowledge and skills include dynamic testing knowledge,
  static testing knowledge, specific tool usage etc.  You should look to
  identify at least one item for each team member.
  \item For each of the knowledge areas and skills identified in the previous
  question, what are at least two approaches to acquiring the knowledge or
  mastering the skill?  Of the identified approaches, which will each team
  member pursue, and why did they make this choice?
\end{enumerate}

\end{document}