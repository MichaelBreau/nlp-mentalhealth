\documentclass{article}

\usepackage{float}
\restylefloat{table}

\usepackage{booktabs}

\title{Team Contributions: Rev 0\\\progname}

\author{\authname}

\date{}

%% Comments

\usepackage{color}

\newif\ifcomments\commentstrue %displays comments
%\newif\ifcomments\commentsfalse %so that comments do not display

\ifcomments
\newcommand{\authornote}[3]{\textcolor{#1}{[#3 ---#2]}}
\newcommand{\todo}[1]{\textcolor{red}{[TODO: #1]}}
\else
\newcommand{\authornote}[3]{}
\newcommand{\todo}[1]{}
\fi

\newcommand{\wss}[1]{\authornote{blue}{SS}{#1}} 
\newcommand{\plt}[1]{\authornote{magenta}{TPLT}{#1}} %For explanation of the template
\newcommand{\an}[1]{\authornote{cyan}{Author}{#1}}

%% Common Parts

\newcommand{\progname}{Natural Language Processing for Mental Health Risk Prediction} % PUT YOUR PROGRAM NAME HERE
\newcommand{\authname}{Team 13, The Cognitive Care Crew
\\ Jessica Dawson
\\ Michael Breau
\\ Matthew Curtis
\\ Benjamin Chinnery
\\ Yaruo Tian} % AUTHOR NAMES                  

\usepackage{hyperref}
    \hypersetup{colorlinks=true, linkcolor=blue, citecolor=blue, filecolor=blue,
                urlcolor=blue, unicode=false}
    \urlstyle{same}
                                


\begin{document}

\maketitle

\section{Demo Plans}

The eRisk competition presents three natural language processing tasks built around predicting mental health indicators from a person's social media posts. For our Rev0 demo we plan to showcase an NLP model for each of this year's three tasks.

\section{Meeting Attendance}

\begin{table}[H]
\centering
\begin{tabular}{ll}
\toprule
\textbf{Student} & \textbf{Meetings}\\
\midrule
Total & 10\\
Benjamin Chinnery & 10\\
Micheal Breau & 10\\
Jessica Dawson & 8\\
Matthew Curtis & 6\\
Yaruo Tian & 8\\
\bottomrule
\end{tabular}
\end{table}
Overall, any noticeable discrepancies within meeting numbers since the POC demo can be accredited to the varying responsibilities of each member. As far as core group discussions and supervisor meetings go, the team has nearly perfect attendance, as those are of the highest priority, and any group members who have been absent, have always taken the initiative to have one on one discussions to catch themselves up.

Additional meetings outside of the core functionality are not mandated for all members to attend, as they may consist of ops team integration, or may be more task specific on topics not associated with that specific member. Teammates which cannot attend these non-mandatory briefings are allowed to devote that time however they wish, as will be briefed on any developments at the next core team meeting.


\section{Lecture Attendance}

\begin{table}[H]
\centering
\begin{tabular}{ll}
\toprule
\textbf{Student} & \textbf{Lectures}\\
\midrule
Total & 2\\
Benjamin Chinnery & 2\\
Micheal Breau & 2\\
Jessica Dawson & 0\\
Matthew Curtis & 1\\
Yaruo Tian & 1\\
\bottomrule
\end{tabular}
\end{table}

Unfortunately, not all members have been available for all lectures since the POC demo due to conflicts with other courses and personal responsibilities. These members have always ensured that other group members will be present, and that relevant information will be relayed to them after the lectures.

\section{Commits}

This data represents the number of commits to both the Document Repository and the Private Code Repository since the POC demonstation. 
\begin{table}[H]
\centering
\begin{tabular}{lll}
\toprule
\textbf{Student} & \textbf{Commits} & \textbf{Percent}\\
\midrule
Total & 75 & 100\% \\
Benjamin Chinnery & 8 & 11\%\\
Micheal Breau & 39 & 52\%\\
Jessica Dawson & 3 & 4\%\\
Matthew Curtis & 13 & 17\%\\
Yaruo Tian & 12 & 16\%\\
\bottomrule
\end{tabular}
\end{table}

The data displayed above would tend to illustrate a rather uneven distribution of work amongst team members, but luckily team members have been satisfied with the overall distribution of effort, and these unequal commit numbers boil down to variations within assigned tasks, associated teamwork per task, as well as individual's committing approach. To elaborate on this, team members working a joint task may find themselves needing to commit more frequently to keep branches up to date, rather than members working on individual assignments. Additionally, the number of tasks can be directly corresponding to the type of task itself, for example Michael's work on the continuous integration Pylint checkers required numerous commits to main in order to test functionality.

Finally, certain efforts from team members may not be sufficiently displayed, if they were spending their time experimenting with methodologies that would not end up being merged into main, as well as if they were updating documentation intended for internal stakeholder and supervisor use on the UQAM overleaf repository, as those private documents are not intended to be pushed to main or shared.



\section{Issue Tracker}

\begin{table}[H]
\centering
\begin{tabular}{lll}
\toprule
\textbf{Student} & \textbf{Authored (O+C)} & \textbf{Assigned (C only)}\\
\midrule
Benjamin Chinnery & 10 & 4 \\
Jessica Dawson& 9 & 2 \\
Matthew Curtis & 10 & 2 \\
Michael Breau & 15 & 9 \\
Yaruo Tian & 18 & 11 \\
\bottomrule
\end{tabular}
\end{table}

These numbers are determined using the issue tracker's filter function.

\section{CICD}

\wss{Say how CICD is used in your project}

\end{document}