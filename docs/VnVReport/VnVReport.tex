\documentclass[12pt, titlepage]{article}

\usepackage{booktabs}
\usepackage{tabularx}
\usepackage{hyperref}
\hypersetup{
    colorlinks,
    citecolor=black,
    filecolor=black,
    linkcolor=red,
    urlcolor=blue
}
\usepackage[round]{natbib}
\usepackage{pdflscape}
\usepackage{longtable}

\input{../Comments}
%% Common Parts

\newcommand{\progname}{Natural Language Processing for Mental Health Risk Prediction} % PUT YOUR PROGRAM NAME HERE
\newcommand{\authname}{Team 13, The Cognitive Care Crew
\\ Jessica Dawson
\\ Michael Breau
\\ Matthew Curtis
\\ Benjamin Chinnery
\\ Yaruo Tian} % AUTHOR NAMES                  

\usepackage{hyperref}
    \hypersetup{colorlinks=true, linkcolor=blue, citecolor=blue, filecolor=blue,
                urlcolor=blue, unicode=false}
    \urlstyle{same}
                                


\begin{document}

\title{Verification and Validation Report: \progname} 
\author{\authname}
\date{\today}
	
\maketitle

\pagenumbering{roman}

\section{Revision History}

\begin{tabularx}{\textwidth}{p{3cm}p{2cm}X}
\toprule {\bf Date} & {\bf Version} & {\bf Notes}\\
\midrule
Date 1 & 1.0 & Notes\\
Date 2 & 1.1 & Notes\\
\bottomrule
\end{tabularx}

~\newpage

\section{Symbols, Abbreviations and Acronyms}

\renewcommand{\arraystretch}{1.2}
\begin{tabular}{l l} 
  \toprule		
  \textbf{symbol} & \textbf{description}\\
  \midrule 
  T & Test\\
  \bottomrule
\end{tabular}\\

\wss{symbols, abbreviations or acronyms -- you can reference the SRS tables if needed}

\newpage

\tableofcontents

\listoftables %if appropriate

\listoffigures %if appropriate

\newpage

\pagenumbering{arabic}

This document ...

\section{Functional Requirements Evaluation}

 \subsection{Task 3}
 \begin{enumerate}
 
 \item \textbf{FRT-T3-1}

Control: Manual

Initial State: The task 3 model has been trained on a section of the training data provided by eRisk

Input: A section of the training data that the model has not yet seen

Output: The system will output predictions to a txt file where each line is of the format "\{Username\} \{prediction for Q1\} ... \{prediction for Q28\}"

How test will be performed: A tester will run the system on the input data and verify that an output file is created and that it is of the correct format

Result: Passed (due to privacy concerns the output text file can not be included here for verification)

\item \textbf{FRT-T3-2}

Control: Manual

Initial State: The task 3 model has been trained on a section of the training data provided by eRisk

Input: A section of the training data that the model has not yet seen

Output: The system will output predictions to a txt file where each prediction should be between 0 and 6

How test will be performed: A tester will run the system on the input data and verify that none of the predictions in the output file are less than 0 or greater than 6

Result: Passed (due to privacy concerns the output text file can not be included here for verification)

\item \textbf{FRT-T3-3}

For test FRT-T3-3, FRT-T3-4, and FRT-T3-5 the system produced the following results:

\begin{tabular}{c|ccccccc}
  &  MAE   & MZOE  & RS    & ECS   & SCS   & WCS   & GED \\ \hline
zeros  & 3.58  & 0.88  & 3.63  & 2.93  & 4.04  & 3.64  & 3.45 \\
sixes  & 2.42  & \textbf{0.65}  & 3.13  & 3.68  & 2.91  & 2.91  & 3.02 \\
average & \textbf{2.00}  & 0.90  & \textbf{1.63}  & 1.59  & 2.04  & \textbf{1.65}  & \textbf{1.47} \\
model  & 2.60  & 0.82  & 2.45  & \textbf{1.47}  & \textbf{1.98}  & 1.79  & 1.59 \\
\end{tabular}

Control: Manual

Initial State: The task 3 model has been trained on a section of the training data provided by eRisk and accuracy measures have been calculated for a baseline strategy where a score of 0 is guessed for every question

Input: A section of the training data that the model has not yet seen

Output: The system will print accuracy metrics for the model to the console where the model scores higher than the baseline strategy on at least one metric

How test will be performed: A tester will run the system on the input data and verify that the model outperforms the baseline on at least one metric

Result: Passed

\item \textbf{FRT-T3-4}

Control: Manual

Initial State: The task 3 model has been trained on a section of the training data provided by eRisk and accuracy measures have been calculated for a baseline strategy where a score of 6 is guessed for every question

Input: A section of the training data that the model has not yet seen

Output: The system will print accuracy metrics for the model to the console where the model scores higher than the baseline strategy on at least one metric

How test will be performed: A tester will run the system on the input data and verify that the model outperforms the baseline on at least one metric

Result: Passed

\item \textbf{FRT-T3-5}

Control: Manual

Initial State: The task 3 model has been trained on a section of the training data provided by eRisk and accuracy measures have been calculated for a baseline strategy where the average score for a question based on the training data is guessed

Input: A section of the training data that the model has not yet seen

Output: The system will print accuracy metrics for the model to the console where the model scores higher than the baseline strategy on at least one metric

How test will be performed: A tester will run the system on the input data and verify that the model outperforms the baseline on at least one metric

Result: Passed

 \end{enumerate}

\section{Nonfunctional Requirements Evaluation}

\subsection{Usability}
		
\subsection{Performance}

\subsection{etc.}
	
\section{Comparison to Existing Implementation}	

\subsection{Task 3}
A comparison between our model and the best metric scores from last year's iteration of the task are compared. The best scores are the highest accuracy achieved on that metric accross all teams, not the metrics from a "best" model.

\begin{tabular}{c|ccccccc}
  &  MAE   & MZOE  & RS    & ECS   & SCS   & WCS   & GED \\ \hline
our model  & 2.60  & 0.82  & 2.45  & \textbf{1.47}  & 1.98  & \textbf{1.79}  & \textbf{1.59} \\
2023's best  & \textbf{2.19}  & \textbf{0.67}  & \textbf{1.59}  & 1.92  & \textbf{1.90}  & 2.00  & 2.00 \\
\end{tabular}

\subsubsection{Discussion}
The model does manage to outperform previous years on the ECS, WCS, and GED metrics. While we fail to outperform MZOE and RS these metrics were achieved by baselines like guessing all zero, sixes, or the average last year, and given we failed to outperform these baselines on these metrics again this year it is possible this has more to do with the nature of the data and the task itself than our model under performing compared to previous year's models. We only failed to outperform previous year's models on MAE and SCS.

\section{Unit Testing}

\section{Changes Due to Testing}

\wss{This section should highlight how feedback from the users and from 
the supervisor (when one exists) shaped the final product.  In particular 
the feedback from the Rev 0 demo to the supervisor (or to potential users) 
should be highlighted.}

\section{Automated Testing}
		
\section{Trace to Requirements}
		
\begin{landscape}
\section{Trace to Requirements}
		\subsection{Traceability Between Test Cases and Requirements}

		Requirement traceability from S2 of SRS to testing in this document.

		\begin{longtable}{|l|cccccccccccccccc|}
			\caption{Traceability Between Functional Test Cases and Functional Requirements, T1FR-1 to T3FR-3}                                                                                                                                                                                                                           \\
			\hline
			\textbf{Test IDs}   & \multicolumn{11}{c|}{\textbf{Functional Requirement IDs}}                                                                                                                                                                                                                                         \\
			\hline
			~                   & \textbf{T1FR-1}                                              & \textbf{T1FR-2} & \textbf{T1FR-3} & \textbf{T2FR-1} & \textbf{T2FR-2} & \textbf{T2FR-3} & \textbf{T2FR-4} & \textbf{T3FR-1} & \textbf{T3FR-2} & \textbf{T3FR-3}  \\
			\hline
   		\textbf{FRT-INP-1}  & X                                                         & ~            & ~            & X            & ~            & ~            & ~            & X            & ~            & ~             \\
        	\textbf{FRT-INP-2}  & X                                                         & ~            & ~            & X            & ~            & ~            & ~            & X            & ~            & ~             \\
			\textbf{FRT-T1-1}  & ~                                                         & X            & ~            & ~            & ~            & ~            & ~            & ~            & ~            & ~             \\
      	\textbf{FRT-T1-2}  & ~                                                         & ~            & X            & ~            & ~            & ~            & ~            & ~            & ~            & ~             \\
            \textbf{FRT-T2-1}  & ~                                                         & ~            & ~            & ~            & X            & ~            & X            & ~            & ~            & ~             \\
            \textbf{FRT-T2-2}  & ~                                                         & ~            & ~            & ~            & ~            & X            & ~            & ~            & ~            & ~             \\
            \textbf{FRT-T2-1}  & ~                                                         & ~            & ~            & ~            & ~            & ~            & ~            & ~            & X            & X             \\     
            \textbf{FRT-T3-2}  & ~                                                         & ~            & ~            & ~            & ~            & ~            & ~            & ~            & X            & X             \\     
            \textbf{FRT-T3-3}  & ~                                                         & ~            & ~            & ~            & ~            & ~            & ~            & ~            & ~            & ~             \\     
            \textbf{FRT-T3-4}  & ~                                                         & ~            & ~            & ~            & ~            & ~            & ~            & ~            & ~            & ~             \\     
            \textbf{FRT-T3-5}  & ~                                                         & ~            & ~            & ~            & ~            & ~            & ~            & ~            & ~            & ~             \\     
			\hline
		\end{longtable}

  \begin{longtable}{|l|ccc|}
			\caption{Traceability Between Functional Test Cases and Functional Requirements, T3FR-4 to T3FR-6}                                                                                                                                                                                                                           \\
			\hline
			\textbf{Test IDs}   & \multicolumn{3}{c|}{\textbf{Functional Requirement IDs}}                                                                                                                                                                                                                                         \\
			\hline
			~                   & \textbf{T3FR-4} & \textbf{T3FR-5} & \textbf{T3FR-6}  \\
			\hline
   		\textbf{FRT-INP-1}  & ~ & ~ & ~ \\
        	\textbf{FRT-INP-2}  & ~ & ~ & ~ \\
          \textbf{FRT-T1-1}  & ~ & ~ & ~ \\
          \textbf{FRT-T1-2}  & ~ & ~ & ~ \\
      	\textbf{FRT-T1-3}  & ~ & ~ & ~ \\
            \textbf{FRT-T2-1} & ~ & ~ & ~ \\
            \textbf{FRT-T2-2} & ~ & ~ & ~ \\
            \textbf{FRT-T2-1} & ~ & ~ & ~ \\     
            \textbf{FRT-T3-2} & ~ & ~ & ~ \\     
            \textbf{FRT-T3-3} & X & ~ & ~ \\     
            \textbf{FRT-T3-4} & ~ & X & ~ \\     
            \textbf{FRT-T3-5} & ~ & ~ & X \\     
			\hline
			
		\end{longtable}

		\begin{longtable}{|l|cccccccccccccccc|}
			\caption{Traceability Between Non-Functional Test Cases and Non-Functional Requirements}                                                                                                                                                                                                                           \\
			\hline
			\textbf{Test IDs}   & \multicolumn{11}{c|}{\textbf{Non-Functional Requirement IDs}}                                                                                                                                                                                                                                         \\
			\hline
			~                   & \textbf{GR1}                                              & \textbf{GR2} & \textbf{SR1} & \textbf{SR2} \\
			\hline
			\textbf{NFRT-U-1}  & X                                                         & ~            & ~            & ~             \\
			\textbf{NFRT-SS-1}  & ~                                                         & ~            & X            & ~            \\
			\textbf{NFRT-SS-2}  & ~                                                         & ~            & ~            & X            \\
			\textbf{NFRT-L-1}  & ~                                                         & X            & ~            & ~             \\
			\hline
		\end{longtable}
\end{landscape}

\section{Traceability of Test Cases and Modules}
\begin{longtable}{|l|ccccccccc|}
    \hline
    \textbf{Test IDs}  & \multicolumn{9}{c|}{\textbf{Module IDs}}                                               \\
    \hline
    ~                  & \textbf{M2}                              & \textbf{M3} & \textbf{EDM1} & \textbf{EDM2} & \textbf{EDM3} & \textbf{EDM4} & \textbf{EDM5} & \textbf{EDM6} & \textbf{EDM7} \\
    \hline
    \textbf{FRT-INP-1}  & X                                        & X           & X             & X             & ~ & ~ & ~ & ~ & ~\\
    \textbf{FRT-INP-2}  & X                                        & X           & X             & X             & ~ & ~ & ~ & ~ & ~\\
    \textbf{FRT-T1-1}  & X                                        & ~           & ~             & ~             & ~ & ~ & ~ & ~ & ~\\
    \textbf{FRT-T1-2}  & X                                        & ~           & ~             & ~             & ~ & ~ & ~ & ~ & ~\\
    \textbf{FRT-T1-3}  & X                                        & ~           & ~             & ~             & ~ & ~ & ~ & ~ & ~\\
    \textbf{FRT-T2-1}  & ~                                        & X           & ~             & ~             & ~ & ~ & ~ & ~ & ~\\
    \textbf{FRT-T2-2}  & ~                                        & X           & ~             & ~             & ~ & ~ & ~ & ~ & ~\\
    \textbf{FRT-T3-1}  & ~                                        & ~           & X             & X             & ~ & ~ & ~ & ~ & ~\\
    \textbf{FRT-T3-2}  & ~                                        & ~           & X             & X             & ~ & ~ & ~ & ~ & ~\\
    \textbf{FRT-T3-3}  & ~                                        & ~           & ~             & ~             & X & X & X & X & X\\
    \textbf{FRT-T3-4}  & ~                                        & ~           & ~             & ~             & X & X & X & X & X\\
    \textbf{FRT-T3-5}  & ~                                        & ~           & ~             & ~             & X & X & X & X & X\\
    \hline
\end{longtable}

\section{Code Coverage Metrics}

\bibliographystyle{plainnat}
\bibliography{../../refs/References}

\newpage{}
\section*{Appendix --- Reflection}

The information in this section will be used to evaluate the team members on the
graduate attribute of Reflection.  Please answer the following question:

\begin{enumerate}
  \item In what ways was the Verification and Validation (VnV) Plan different
  from the activities that were actually conducted for VnV?  If there were
  differences, what changes required the modification in the plan?  Why did
  these changes occur?  Would you be able to anticipate these changes in future
  projects?  If there weren't any differences, how was your team able to clearly
  predict a feasible amount of effort and the right tasks needed to build the
  evidence that demonstrates the required quality?  (It is expected that most
  teams will have had to deviate from their original VnV Plan.)
\end{enumerate}

\end{document}