\documentclass[12pt, titlepage]{article}

\usepackage{booktabs}
\usepackage{tabularx}
\usepackage{hyperref}
\hypersetup{
    colorlinks,
    citecolor=black,
    filecolor=black,
    linkcolor=red,
    urlcolor=blue
}
\usepackage[round]{natbib}
\usepackage{pdflscape}
\usepackage{longtable}

%% Comments

\usepackage{color}

\newif\ifcomments\commentstrue %displays comments
%\newif\ifcomments\commentsfalse %so that comments do not display

\ifcomments
\newcommand{\authornote}[3]{\textcolor{#1}{[#3 ---#2]}}
\newcommand{\todo}[1]{\textcolor{red}{[TODO: #1]}}
\else
\newcommand{\authornote}[3]{}
\newcommand{\todo}[1]{}
\fi

\newcommand{\wss}[1]{\authornote{blue}{SS}{#1}} 
\newcommand{\plt}[1]{\authornote{magenta}{TPLT}{#1}} %For explanation of the template
\newcommand{\an}[1]{\authornote{cyan}{Author}{#1}}

%% Common Parts

\newcommand{\progname}{Natural Language Processing for Mental Health Risk Prediction} % PUT YOUR PROGRAM NAME HERE
\newcommand{\authname}{Team 13, The Cognitive Care Crew
\\ Jessica Dawson
\\ Michael Breau
\\ Matthew Curtis
\\ Benjamin Chinnery
\\ Yaruo Tian} % AUTHOR NAMES                  

\usepackage{hyperref}
    \hypersetup{colorlinks=true, linkcolor=blue, citecolor=blue, filecolor=blue,
                urlcolor=blue, unicode=false}
    \urlstyle{same}
                                


\begin{document}

\title{Verification and Validation Report: \progname} 
\author{\authname}
\date{\today}
	
\maketitle

\pagenumbering{roman}

\section{Revision History}

\begin{tabularx}{\textwidth}{p{3cm}p{2cm}X}
\toprule {\bf Date} & {\bf Version} & {\bf Notes}\\
\midrule
3/6/2024 & 1.0 & Revision 0\\
\bottomrule
\end{tabularx}

~\newpage

\section{Symbols, Abbreviations and Acronyms}

\renewcommand{\arraystretch}{1.2}
\begin{tabular}{l l} 
  \toprule		
  \textbf{symbol} & \textbf{description}\\
  \midrule 
  Precision & the proportion of positive guesses that were correct\\
  Recall & the proportion of positive individuals that were correctly predicted\\
  F-score & combination of precision and recall\\
  R-PREC & represents balancing recall and precision\\
  NDCG & measures the effectiveness of a ranked list in information retrieval\\
  RMSE & Root Mean Square Error\\
  MZOE & Mean Zero-One Error\\
  MAE & Mean Absolute Error\\
  RS & Restraint Subscale\\
  ECS & Eating Concern Subscale\\
  SCS & Shape Concern Subscale\\
  WCS & Weight Concern Subscale\\
  \bottomrule
\end{tabular}\\

\newpage

\tableofcontents

\listoftables %if appropriate

\listoffigures %if appropriate

\newpage

\pagenumbering{arabic}


\section{Functional Requirements Evaluation}
\subsection{Task 1}

\begin{enumerate}
\item \textbf{FRT-T1-1}

Control: Automatic

Initial State: No models have been trained by the task 1 system yet, input data as jsonl files are available

Input: The Task 1 Input data

Output: The system will extract required sentences from the jsonl files and output data that is appropriate to be fed into Task 1.

Test Case Derivation: The system will receive its input data, process then parse according to the Task's input format.

How test will be performed: The tester will run the parser with the given jsonl files as input and compare the parsed data.

Results: Passed

\item \textbf{FRT-T1-2}

Control: Automatic

Initial State: No models have been trained by the task 1 system yet, past years input data, training data, and golden truth values are available

Input: The Task 1 Training Data, Input data, and Golden Truth Values from a past year eRisk Competition

Output: The system will output sentence ranking and diagnostic metrics based on the input data. 

Test Case Derivation: The system will receive its training data and input data corresponding to a prior year. It's analysis will output RP and NDCG accuracy metrics.

How test will be performed: The system should derive its training data and feature sets based on correlation to 21 depression symptoms of beck depression index. The outputted metrics will be compared to golden truth data to ensure the accuracy is within the desired bounds and is improving on prior year implementation.

Results: Passed

\item \textbf{FRT-T1-3}

Control: Automatic

Initial State: No models have been trained by the task 1 system yet, input data is available

Input: The Task 1 Input data

Output: The system will output sentence ranking and diagnostic metrics based on the input data to a txt file.

Test Case Derivation: The system will make an analysis based off the provided input data and return each prediction to a test file in a specified formatting.

How test will be performed: The system's output txt file will be verified that each entry is on its own line in the format "\{symptom\_number\}, Q0, \{sentence-id\}, \{position\_in\_ranking\}, \{score\}, \{system\_name\}".

Results: Passed

\end{enumerate}

\subsection{Task 2}

 \begin{enumerate}
 
 \item \textbf{FRT-T2-1}

Control: Automatic

Initial State: No models have been trained by the task 2 system yet, input data is available

Input: The Task 2 Input data consisting of user posts

Output: The system will output sentence ranking and diagnostic metrics based on the input data to a txt file.

Test Case Derivation: The system will make an analysis based off the provided input data and return each prediction to a test file in a specified formatting.

How test will be performed: The system's output txt file will be verified that each entry is on its own line in the format "\{Username\} \{prediction\}".

Results: Passed

 \item \textbf{FRT-T2-2}
 Control: Automatic

Initial State: The task 2 system has received input data on a given individual and made a corresponding prediction.

Input: More data corresponding to the same user analyzed in the input state.

Output: The system will output an updated prediction taking into account all data provided.

Test Case Derivation: The system should create a new prediction for the chosen user taking into account all data provided.

How test will be performed: The system should provide a prediction based off the original data input, after receiving new data, the system should combine posts made by the same user to create a new prediction. 

Results: Passed
 \end{enumerate}

 \subsection{Task 3}
 \begin{enumerate}
 
 \item \textbf{FRT-T3-1}

Control: Manual

Initial State: The task 3 model has been trained on a section of the training data provided by eRisk

Input: A section of the training data that the model has not yet seen

Output: The system will output predictions to a txt file where each line is of the format "\{Username\} \{prediction for Q1\} ... \{prediction for Q28\}"

How test will be performed: A tester will run the system on the input data and verify that an output file is created and that it is of the correct format

Result: Passed (due to privacy concerns the output text file can not be included here for verification)

\item \textbf{FRT-T3-2}

Control: Manual

Initial State: The task 3 model has been trained on a section of the training data provided by eRisk

Input: A section of the training data that the model has not yet seen

Output: The system will output predictions to a txt file where each prediction should be between 0 and 6

How test will be performed: A tester will run the system on the input data and verify that none of the predictions in the output file are less than 0 or greater than 6

Result: Passed (due to privacy concerns the output text file can not be included here for verification)

\item \textbf{FRT-T3-3}

For test FRT-T3-3, FRT-T3-4, and FRT-T3-5 the system produced the following results:

\begin{tabular}{c|ccccccc}
  &  MAE   & MZOE  & RS    & ECS   & SCS   & WCS   & GED \\ \hline
zeros  & 3.58  & 0.88  & 3.63  & 2.93  & 4.04  & 3.64  & 3.45 \\
sixes  & 2.42  & \textbf{0.65}  & 3.13  & 3.68  & 2.91  & 2.91  & 3.02 \\
average & \textbf{2.00}  & 0.90  & \textbf{1.63}  & 1.59  & 2.04  & \textbf{1.65}  & \textbf{1.47} \\
model  & 2.60  & 0.82  & 2.45  & \textbf{1.47}  & \textbf{1.98}  & 1.79  & 1.59 \\
\end{tabular}

Control: Manual

Initial State: The task 3 model has been trained on a section of the training data provided by eRisk and accuracy measures have been calculated for a baseline strategy where a score of 0 is guessed for every question

Input: A section of the training data that the model has not yet seen

Output: The system will print accuracy metrics for the model to the console where the model scores higher than the baseline strategy on at least one metric

How test will be performed: A tester will run the system on the input data and verify that the model outperforms the baseline on at least one metric

Result: Passed

\item \textbf{FRT-T3-4}

Control: Manual

Initial State: The task 3 model has been trained on a section of the training data provided by eRisk and accuracy measures have been calculated for a baseline strategy where a score of 6 is guessed for every question

Input: A section of the training data that the model has not yet seen

Output: The system will print accuracy metrics for the model to the console where the model scores higher than the baseline strategy on at least one metric

How test will be performed: A tester will run the system on the input data and verify that the model outperforms the baseline on at least one metric

Result: Passed

\item \textbf{FRT-T3-5}

Control: Manual

Initial State: The task 3 model has been trained on a section of the training data provided by eRisk and accuracy measures have been calculated for a baseline strategy where the average score for a question based on the training data is guessed

Input: A section of the training data that the model has not yet seen

Output: The system will print accuracy metrics for the model to the console where the model scores higher than the baseline strategy on at least one metric

How test will be performed: A tester will run the system on the input data and verify that the model outperforms the baseline on at least one metric

Result: Passed

 \end{enumerate}

\section{Nonfunctional Requirements Evaluation}

\subsection{Usability Requirements}

\begin{enumerate}

\item \textbf{NFRT-U-1}
    
Type: Dynamic, Manual
    					
Initial State: System is completed and trained
    					
Input/Condition: The system will be ran on a Windows desktop/laptop and macOS desktop/laptop device with the correct environment setup
    					
Output/Result: The system should generate an expected result 
    					
How test will be performed: After setting up the environment on to both macOS and Windows machines, the system will be ran on both environment with the same command. Test will pass as long as an expected result is generated from both machines.

Results: Passed
    
\end{enumerate}
		
\subsection{Safety and Security Requirements}

\begin{enumerate}

\item \textbf{NFRT-SS-1}

Type: Dynamic, Manual
					
Initial State: The system is completed and trained 
					
Input/Condition: The system will be ran on the developer's computer
					
Output/Result: The generated result should not overuse the processing power of the developer's computer
					
How test will be performed: CPU, GPU, RAM and live power usage will be monitored while running the software. Usage of all aspects should not increase more than 20 percent.   

Results: Passed
					
\end{enumerate}


\begin{enumerate}

\item \textbf{NFRT-SS-2}
	
Type: Automatic, Dynamic
						
Initial State: The system has been trained and is awaiting data to form predictions on.
						
Input/Condition: A set of data that the system can predict on.
						
Output/Result: The resulting predictions, in a form where no sensitive data from the input is present in the output.
	
Test Case Derivation: Sensitive in this case refers to the post history of the subjects eRisk provides as training and test data. If these posts can be reconstructed from any part of our system outputs it is possible the identity of the individual could be discovered. This represents an unacceptable breach of privacy and can not happen.
						
How test will be performed: After the system has produced it's predictions a script will be run that takes all posts in the input data, forms a string out of all consecutive three word triplets (ie. "hello my good friend" forms "hello my good" and "my good friend"), both with and without processing (removal of stopwords, punctuation, etc.), and scans the output to ensure that none of these triples are present.

Results: Passed
	
\end{enumerate}

\subsection{Legal Requirements}

\begin{enumerate}

\item \textbf{NFRT-L-1}

Type: Static, Manual
					
Initial State: The source code and documentation is prepared
					
Input/Condition: The user reviews the entirety of the source code and related documentation
					
Output/Result: Copyright licenses, appropriate credits and/or MIT license must attached
					
How test will be performed: The testers will review the entirely of the project and check to see if appropriate copyright licenses and/or credits are included for resources that require them

Results: Passed

\end{enumerate}

	
\section{Comparison to Existing Implementation}	

A large portion of our validation is deploying our models for the eRisk 2024 competition to determine how well they perform compared to models from other participants. As the current competition is not over, the metrics to compare against were derived from the performances of teams from previous year eRisk competitions.

\subsection{Task 1}
Metrics were compared against eRisk's 2022 Task 1: Search for symptoms of depression

\subsubsection{Our Model}
\begin{itemize}
\item R-PREC: 0.219
\item NDCG: 0.683
\end{itemize}

\subsubsection{Best Performing Model from Previous Years}
\begin{itemize}
\item R-PREC: 0.375
\item NDCG: 0.596
\end{itemize}

\subsubsection{Discussion}
In comparison to the accuracy metric of the 2022's best performing model for this task, which were 0.375 for R-PREC and 0.596 for NDCG. We outperformed it in terms of the NDCG metric with a value of 0.683 but lacking in R-PREC with a value of 0.219. The greater NDCG, which shows better performance in rating relevant phrases despite the lower R-PREC, highlights how well the system prioritizes the quality of relevant sentence placement. The best metric to use will rely on the particular objectives of the NLP work connected to depression, taking into account the trade-off between recall and precision. Although the recall precision score from the prior year was greater, indicating a better balance, our system performs very well in NDCG, emphasizing the importance of quality ranking for recognizing important sentences.


\subsection{Task 2}
Metrics were compared against eRisk's 2021 Task 1: Early Detection of Pathological Gambling

\subsubsection{Our Model}
\begin{itemize}
    \item Precision: 0.627
    \item Recall: 0.644
    \item F-score: 0.635
\end{itemize}

\subsubsection{Best Performing Model from Previous Years}
\begin{itemize}
    \item Precision: 0.586
    \item Recall: 0.939
    \item F-score: 0.721
\end{itemize}

\subsubsection{Discussion}
Overall F-score of our model was slightly lower than the best performing model from the previous year, our model is lacking heavily compared to other models in the recall section but has the best overall precision of any model and would place second out of the six teams during that eRisk competition. It is important to note that this was early risk prediction for gambling addiction whereas ours is for anorexia and therefore cannot be directly compared and should just be used as overall reference for the broad category of positive negative classification.

\subsection{Task 3}
A comparison between our model and the best metric scores from last year's iteration of the task are compared. The best scores are the highest accuracy achieved on that metric accross all teams, not the metrics from a "best" model.

\begin{tabular}{c|ccccccc}
  &  MAE   & MZOE  & RS    & ECS   & SCS   & WCS   & GED \\ \hline
our model  & 2.60  & 0.82  & 2.45  & \textbf{1.47}  & 1.98  & \textbf{1.79}  & \textbf{1.59} \\
2023's best  & \textbf{2.19}  & \textbf{0.67}  & \textbf{1.59}  & 1.92  & \textbf{1.90}  & 2.00  & 2.00 \\
\end{tabular}

\subsubsection{Discussion}
The model does manage to outperform previous years on the ECS, WCS, and GED metrics. While we fail to outperform MZOE and RS these metrics were achieved by baselines like guessing all zero, sixes, or the average last year, and given we failed to outperform these baselines on these metrics again this year it is possible this has more to do with the nature of the data and the task itself than our model under performing compared to previous year's models. We only failed to outperform previous year's models on MAE and SCS.


\section{Unit Testing}

Due to the nature of our project being almost entirely output focused. Our
tests do not directly test the individual modules as the code is verified directly by the
fact that the metrics of the output are directly showing that the model is
performing as intended. Any error in individual units would produce incorrect output or error and as a result was deemed to not be necessary. 

\section{Changes Due to Testing}

The supervisors for this project as individuals who had participated in previous competitions were instrumental in the understanding of the project and bi-weekly update meetings led to iterative improvements to the methodology of the choices in our models as well as the performance in regards to metrics.

\section{Automated Testing}

Automated testing for each individual task was done using pytest by analyzing the output and checking for correct output and checking that the models are running as intended. To do this the test module calculates the metrics of the results and checks if the resulting metrics are above a certain threshold where if it was below this threshold, something is most likely going wrong with the model.


\begin{landscape}
\section{Trace to Requirements}
		\subsection{Traceability Between Test Cases and Requirements}

		Requirement traceability from S2 of SRS to testing in this document.

		\begin{longtable}{|l|cccccccccccccccc|}
			\caption{Traceability Between Functional Test Cases and Functional Requirements, T1FR-1 to T3FR-3}                                                                                                                                                                                                                           \\
			\hline
			\textbf{Test IDs}   & \multicolumn{11}{c|}{\textbf{Functional Requirement IDs}}                                                                                                                                                                                                                                         \\
			\hline
			~                   & \textbf{T1FR-1}                                              & \textbf{T1FR-2} & \textbf{T1FR-3} & \textbf{T2FR-1} & \textbf{T2FR-2} & \textbf{T2FR-3} & \textbf{T2FR-4} & \textbf{T3FR-1} & \textbf{T3FR-2} & \textbf{T3FR-3}  \\
			\hline
   		\textbf{FRT-INP-1}  & X                                                         & ~            & ~            & X            & ~            & ~            & ~            & X            & ~            & ~             \\
        	\textbf{FRT-INP-2}  & X                                                         & ~            & ~            & X            & ~            & ~            & ~            & X            & ~            & ~             \\
			\textbf{FRT-T1-1}  & ~                                                         & X            & ~            & ~            & ~            & ~            & ~            & ~            & ~            & ~             \\
      	\textbf{FRT-T1-2}  & ~                                                         & ~            & X            & ~            & ~            & ~            & ~            & ~            & ~            & ~             \\
            \textbf{FRT-T2-1}  & ~                                                         & ~            & ~            & ~            & X            & ~            & X            & ~            & ~            & ~             \\
            \textbf{FRT-T2-2}  & ~                                                         & ~            & ~            & ~            & ~            & X            & ~            & ~            & ~            & ~             \\
            \textbf{FRT-T3-1}  & ~                                                         & ~            & ~            & ~            & ~            & ~            & ~            & ~            & X            & X             \\     
            \textbf{FRT-T3-2}  & ~                                                         & ~            & ~            & ~            & ~            & ~            & ~            & ~            & X            & X             \\     
            \textbf{FRT-T3-3}  & ~                                                         & ~            & ~            & ~            & ~            & ~            & ~            & ~            & ~            & ~             \\     
            \textbf{FRT-T3-4}  & ~                                                         & ~            & ~            & ~            & ~            & ~            & ~            & ~            & ~            & ~             \\     
            \textbf{FRT-T3-5}  & ~                                                         & ~            & ~            & ~            & ~            & ~            & ~            & ~            & ~            & ~             \\     
			\hline
		\end{longtable}

  \begin{longtable}{|l|ccc|}
			\caption{Traceability Between Functional Test Cases and Functional Requirements, T3FR-4 to T3FR-6}                                                                                                                                                                                                                           \\
			\hline
			\textbf{Test IDs}   & \multicolumn{3}{c|}{\textbf{Functional Requirement IDs}}                                                                                                                                                                                                                                         \\
			\hline
			~                   & \textbf{T3FR-4} & \textbf{T3FR-5} & \textbf{T3FR-6}  \\
			\hline
   		\textbf{FRT-INP-1}  & ~ & ~ & ~ \\
        	\textbf{FRT-INP-2}  & ~ & ~ & ~ \\
          \textbf{FRT-T1-1}  & ~ & ~ & ~ \\
          \textbf{FRT-T1-2}  & ~ & ~ & ~ \\
      	\textbf{FRT-T1-3}  & ~ & ~ & ~ \\
            \textbf{FRT-T2-1} & ~ & ~ & ~ \\
            \textbf{FRT-T2-2} & ~ & ~ & ~ \\
            \textbf{FRT-T2-1} & ~ & ~ & ~ \\     
            \textbf{FRT-T3-2} & ~ & ~ & ~ \\     
            \textbf{FRT-T3-3} & X & ~ & ~ \\     
            \textbf{FRT-T3-4} & ~ & X & ~ \\     
            \textbf{FRT-T3-5} & ~ & ~ & X \\     
			\hline
			
		\end{longtable}

		\begin{longtable}{|l|cccccccccccccccc|}
			\caption{Traceability Between Non-Functional Test Cases and Non-Functional Requirements}                                                                                                                                                                                                                           \\
			\hline
			\textbf{Test IDs}   & \multicolumn{11}{c|}{\textbf{Non-Functional Requirement IDs}}                                                                                                                                                                                                                                         \\
			\hline
			~                   & \textbf{GR1}                                              & \textbf{GR2} & \textbf{SR1} & \textbf{SR2} \\
			\hline
			\textbf{NFRT-U-1}  & X                                                         & ~            & ~            & ~             \\
			\textbf{NFRT-SS-1}  & ~                                                         & ~            & X            & ~            \\
			\textbf{NFRT-SS-2}  & ~                                                         & ~            & ~            & X            \\
			\textbf{NFRT-L-1}  & ~                                                         & X            & ~            & ~             \\
			\hline
		\end{longtable}
\end{landscape}

\section{Traceability of Test Cases and Modules}
\begin{longtable}{|l|ccccccccc|}
    \hline
    \textbf{Test IDs}  & \multicolumn{9}{c|}{\textbf{Module IDs}}                                               \\
    \hline
    ~                  & \textbf{M2}                              & \textbf{M3} & \textbf{EDM1} & \textbf{EDM2} & \textbf{EDM3} & \textbf{EDM4} & \textbf{EDM5} & \textbf{EDM6} & \textbf{EDM7} \\
    \hline
    \textbf{FRT-INP-1}  & X                                        & X           & X             & X             & ~ & ~ & ~ & ~ & ~\\
    \textbf{FRT-INP-2}  & X                                        & X           & X             & X             & ~ & ~ & ~ & ~ & ~\\
    \textbf{FRT-T1-1}  & X                                        & ~           & ~             & ~             & ~ & ~ & ~ & ~ & ~\\
    \textbf{FRT-T1-2}  & X                                        & ~           & ~             & ~             & ~ & ~ & ~ & ~ & ~\\
    \textbf{FRT-T1-3}  & X                                        & ~           & ~             & ~             & ~ & ~ & ~ & ~ & ~\\
    \textbf{FRT-T2-1}  & ~                                        & X           & ~             & ~             & ~ & ~ & ~ & ~ & ~\\
    \textbf{FRT-T2-2}  & ~                                        & X           & ~             & ~             & ~ & ~ & ~ & ~ & ~\\
    \textbf{FRT-T3-1}  & ~                                        & ~           & X             & X             & ~ & ~ & ~ & ~ & ~\\
    \textbf{FRT-T3-2}  & ~                                        & ~           & X             & X             & ~ & ~ & ~ & ~ & ~\\
    \textbf{FRT-T3-3}  & ~                                        & ~           & ~             & ~             & X & X & X & X & X\\
    \textbf{FRT-T3-4}  & ~                                        & ~           & ~             & ~             & X & X & X & X & X\\
    \textbf{FRT-T3-5}  & ~                                        & ~           & ~             & ~             & X & X & X & X & X\\
    \hline
\end{longtable}

\section{Code Coverage Metrics}
Due to the nature of our project being almost entirely output focused. Our tests do not directly test the code as the code is verified directly by the fact that the metrics of the output are directly showing that the model is performing as intended.

\bibliographystyle{plainnat}
\bibliography{../../refs/References}

\newpage{}
\section*{Appendix --- Reflection}

The information in this section will be used to evaluate the team members on the
graduate attribute of Reflection.  Please answer the following question:

\begin{enumerate}
  \item In what ways was the Verification and Validation (VnV) Plan different
  from the activities that were actually conducted for VnV?  If there were
  differences, what changes required the modification in the plan?  Why did
  these changes occur?  Would you be able to anticipate these changes in future
  projects?  If there weren't any differences, how was your team able to clearly
  predict a feasible amount of effort and the right tasks needed to build the
  evidence that demonstrates the required quality?  (It is expected that most
  teams will have had to deviate from their original VnV Plan.)
\end{enumerate}

When the team originally created the VnVPlan, our understanding of our project and the eRisk competition requirements were much more lacking at the time and as a result we have had to change a lot of our requirements as well as the tests for those requirements to fit our project. Originally our understanding of the fundamental structure of the NLP pipeline lacked depth and as such our tests did not make much sense after we completed the code for the three tasks. In future projects we can anticipate that we currently do not know the full extent of what we are undertaking so being more clear about what we do not know in our VnV plan would lead to less need for large amount of changes in test structure and content.

\end{document}
