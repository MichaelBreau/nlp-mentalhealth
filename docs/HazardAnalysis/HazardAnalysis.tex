\documentclass{article}

\usepackage{booktabs}
\usepackage{tabularx}
\usepackage{hyperref}

\hypersetup{
    colorlinks=true,       % false: boxed links; true: colored links
    linkcolor=red,          % color of internal links (change box color with linkbordercolor)
    citecolor=green,        % color of links to bibliography
    filecolor=magenta,      % color of file links
    urlcolor=cyan           % color of external links
}

\title{Hazard Analysis\\\progname}

\author{\authname}

\date{}

%% Comments

\usepackage{color}

\newif\ifcomments\commentstrue %displays comments
%\newif\ifcomments\commentsfalse %so that comments do not display

\ifcomments
\newcommand{\authornote}[3]{\textcolor{#1}{[#3 ---#2]}}
\newcommand{\todo}[1]{\textcolor{red}{[TODO: #1]}}
\else
\newcommand{\authornote}[3]{}
\newcommand{\todo}[1]{}
\fi

\newcommand{\wss}[1]{\authornote{blue}{SS}{#1}} 
\newcommand{\plt}[1]{\authornote{magenta}{TPLT}{#1}} %For explanation of the template
\newcommand{\an}[1]{\authornote{cyan}{Author}{#1}}

%% Common Parts

\newcommand{\progname}{Natural Language Processing for Mental Health Risk Prediction} % PUT YOUR PROGRAM NAME HERE
\newcommand{\authname}{Team 13, The Cognitive Care Crew
\\ Jessica Dawson
\\ Michael Breau
\\ Matthew Curtis
\\ Benjamin Chinnery
\\ Yaruo Tian} % AUTHOR NAMES                  

\usepackage{hyperref}
    \hypersetup{colorlinks=true, linkcolor=blue, citecolor=blue, filecolor=blue,
                urlcolor=blue, unicode=false}
    \urlstyle{same}
                                


\begin{document}

\maketitle
\thispagestyle{empty}

~\newpage

\pagenumbering{roman}

\begin{table}[hp]
\caption{Revision History} \label{TblRevisionHistory}
\begin{tabularx}{\textwidth}{llX}
\toprule
\textbf{Date} & \textbf{Developer(s)} & \textbf{Change}\\
\midrule
Date1 & Name(s) & Description of changes\\
Date2 & Name(s) & Description of changes\\
... & ... & ...\\
\bottomrule
\end{tabularx}
\end{table}

~\newpage

\tableofcontents

~\newpage

\pagenumbering{arabic}

\wss{You are free to modify this template.}

\section{Introduction}

\wss{You can include your definition of what a hazard is here.}

\section{Scope and Purpose of Hazard Analysis}

\section{System Boundaries and Components}

\section{Critical Assumptions}

\wss{These assumptions that are made about the software or system.  You should
minimize the number of assumptions that remove potential hazards.  For instance,
you could assume a part will never fail, but it is generally better to include
this potential failure mode.}

\begin{itemize}
    \item We will assume that any health care provider using our product would not intentionally misuse it.
    \item We will assume that any health care provider using our product would not use our product with malicious intent and only use it to help their patient.
    \item We will assume that the libraries and functions we utilize in our code work as expected such as NumPy, Pandas, and SkLearn.
  \end{itemize}


\section{Failure Mode and Effect Analysis}

\wss{Include your FMEA table here}

\section{Safety and Security Requirements}

\wss{Newly discovered requirements.  These should also be added to the SRS.  (A
rationale design process how and why to fake it.)}

SR3. The system will be tested periodically to detect crashes and a potential lack of processing power\\
    \indent \indent \emph{Rationale:} As we build our system out we will be adding more features and increasing the complexity, which increases the processing power required. This may lead to crashes or issues with a lack of processing power that will need to periodically be checked. This is to check the hardware limitations. \\
    \indent \indent \emph{Associated Hazards:} H2-1\\

 SR4. The system will run through a series of tests before deploying/pushing new builds\\
    \indent \indent \emph{Rationale:} This will be done to ensure that the system is still working as expected with our new changes and that nothing has broken before we deploy a new build. These tests will help us detect that and allow us to go back and fix any issues that arise. \\
    \indent \indent \emph{Associated Hazards:} H4-1  \\

SR5. The team will regularly push their code by committing to a repository while working on the code\\
     \indent \indent \emph{Rationale:} This needs to be done in order to avoid losing code/work. If something were to happen to the team members computer for whatever reason and the work that was done after the last commit was to be deleted/lost, frequent commits would help ensure minimal lose.\\
    \indent \indent \emph{Associated Hazards:} H5-1 \\

\section{Roadmap}

\wss{Which safety requirements will be implemented as part of the capstone timeline?
Which requirements will be implemented in the future?}

\end{document}