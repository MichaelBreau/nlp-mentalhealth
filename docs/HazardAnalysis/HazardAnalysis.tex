\documentclass{article}

\usepackage{booktabs}
\usepackage{tabularx}
\usepackage{hyperref}
\usepackage{pdflscape}
\usepackage{float}
\usepackage{adjustbox}
\usepackage{multirow}
\usepackage{enumitem}
\usepackage{longtable}
\usepackage[round]{natbib}
\usepackage[autostyle]{csquotes}


\hypersetup{
    colorlinks=true,       % false: boxed links; true: colored links
    linkcolor=red,          % color of internal links (change box color with linkbordercolor)
    citecolor=green,        % color of links to bibliography
    filecolor=magenta,      % color of file links
    urlcolor=cyan           % color of external links
}

\title{Hazard Analysis\\\progname}

\author{\authname}

\date{}

%% Comments

\usepackage{color}

\newif\ifcomments\commentstrue %displays comments
%\newif\ifcomments\commentsfalse %so that comments do not display

\ifcomments
\newcommand{\authornote}[3]{\textcolor{#1}{[#3 ---#2]}}
\newcommand{\todo}[1]{\textcolor{red}{[TODO: #1]}}
\else
\newcommand{\authornote}[3]{}
\newcommand{\todo}[1]{}
\fi

\newcommand{\wss}[1]{\authornote{blue}{SS}{#1}} 
\newcommand{\plt}[1]{\authornote{magenta}{TPLT}{#1}} %For explanation of the template
\newcommand{\an}[1]{\authornote{cyan}{Author}{#1}}

%% Common Parts

\newcommand{\progname}{Natural Language Processing for Mental Health Risk Prediction} % PUT YOUR PROGRAM NAME HERE
\newcommand{\authname}{Team 13, The Cognitive Care Crew
\\ Jessica Dawson
\\ Michael Breau
\\ Matthew Curtis
\\ Benjamin Chinnery
\\ Yaruo Tian} % AUTHOR NAMES                  

\usepackage{hyperref}
    \hypersetup{colorlinks=true, linkcolor=blue, citecolor=blue, filecolor=blue,
                urlcolor=blue, unicode=false}
    \urlstyle{same}
                                


\begin{document}

\maketitle
\thispagestyle{empty}

~\newpage

\pagenumbering{roman}

\begin{table}[hp]
\caption{Revision History} \label{TblRevisionHistory}
\begin{tabularx}{\textwidth}{llX}
\toprule
\textbf{Date} & \textbf{Developer(s)} & \textbf{Change}\\
\midrule
October 19, 2023 & Jessica Dawson & Adding section 2 and parts of FMEA\\
\bottomrule
\end{tabularx}
\end{table}

~\newpage

\tableofcontents

~\newpage

\pagenumbering{arabic}

\section{Introduction}

\wss{You can include your definition of what a hazard is here.}

\section{Scope and Purpose of Hazard Analysis}

The purpose of this hazard analysis is to identify potential risks to stakeholders and the failure modes of the software that lead to these risks being realized. The document only covers stakeholder losses caused by a failure of our software system as the focus is on guiding the design of the system, not defining all possible stakeholder losses. The document identifies potential hazards in the system and from these derives a set of security requirements that will be used to refine our system design.

\section{System Boundaries and Components}

\section{Critical Assumptions}

\wss{These assumptions that are made about the software or system.  You should
minimize the number of assumptions that remove potential hazards.  For instance,
you could assume a part will never fail, but it is generally better to include
this potential failure mode.}

\begin{landscape}
    \section{Failure Mode and Effect Analysis}
    \begin{longtable}{|p{0.20\textwidth}|p{0.17\textwidth}|p{0.25\textwidth}|p{0.25\textwidth}|p{0.35\textwidth}|p{0.1\textwidth}|p{0.05\textwidth}|}
        \caption{Failure Mode and Effect Analysis Table} \\
        \hline
        \textbf{Component} & \textbf{Failure Modes} & \textbf{Effects of Failure} & \textbf{Causes of Failure} & \textbf{Recommended Action} & \textbf{SR} & \textbf{Ref.} \\
        \hline
        \multirow{2}{*}{General} 
        & An nlp pipeline breaks on deployment
        & Loss of time to try different models and approaches during the eRisk competition
        & Lack of comprehensive validation systems before deployment
        & Implementation of pipeline testing systems to ensure issues are caught early
        & SR4
        & TBD \\
        \cline{2-7}
        & Loss of code
        & Loss of time to try different models and approaches during the eRisk competition
        & Lost code was not backed-up
        & Robust backup systems implemented into development processes
        & SR5
        & TBD \\
        \hline
        \multirow{1}{*}{Risk Prediction} 
        & Corrupted data causes incorrect model results
        & Model gives a wrong prediction, impacts success in the eRisk competition and could result in an incorrect diagnosis
        & Data is distorted in some way (entries duplicated, deleted, changed, etc.)
        & System will check data to ensure it is intact before it is given to the model
        & SR6, SR7
        & TBD \\
        \hline
    \end{longtable}
\end{landscape}

\section{Safety and Security Requirements}

\wss{Newly discovered requirements.  These should also be added to the SRS.  (A
rationale design process how and why to fake it.)}

\section{Roadmap}

\wss{Which safety requirements will be implemented as part of the capstone timeline?
Which requirements will be implemented in the future?}

\end{document}