\documentclass{article}

\usepackage{booktabs}
\usepackage{tabularx}
\usepackage{hyperref}

\hypersetup{
    colorlinks=true,       % false: boxed links; true: colored links
    linkcolor=red,          % color of internal links (change box color with linkbordercolor)
    citecolor=green,        % color of links to bibliography
    filecolor=magenta,      % color of file links
    urlcolor=cyan           % color of external links
}

\title{Hazard Analysis\\\progname}

\author{\authname}

\date{}

\input{../Comments}
%% Common Parts

\newcommand{\progname}{Natural Language Processing for Mental Health Risk Prediction} % PUT YOUR PROGRAM NAME HERE
\newcommand{\authname}{Team 13, The Cognitive Care Crew
\\ Jessica Dawson
\\ Michael Breau
\\ Matthew Curtis
\\ Benjamin Chinnery
\\ Yaruo Tian} % AUTHOR NAMES                  

\usepackage{hyperref}
    \hypersetup{colorlinks=true, linkcolor=blue, citecolor=blue, filecolor=blue,
                urlcolor=blue, unicode=false}
    \urlstyle{same}
                                


\begin{document}

\maketitle
\thispagestyle{empty}

~\newpage

\pagenumbering{roman}

\begin{table}[hp]
\caption{Revision History} \label{TblRevisionHistory}
\begin{tabularx}{\textwidth}{llX}
\toprule
\textbf{Date} & \textbf{Developer(s)} & \textbf{Change}\\
\midrule
Date1 & Name(s) & Description of changes\\
Date2 & Name(s) & Description of changes\\
... & ... & ...\\
\bottomrule
\end{tabularx}
\end{table}

~\newpage

\tableofcontents

~\newpage

\pagenumbering{arabic}

\wss{You are free to modify this template.}

\section{Introduction}

\wss{You can include your definition of what a hazard is here.}

\section{Scope and Purpose of Hazard Analysis}

\section{System Boundaries and Components}

The system that the hazard analysis with be conducted on consists of:
\begin{enumerate}
	\item The system's security/ethical aspects which consist of the following:
	      \begin{itemize}
		      \item Data ingestion component
		      \item Data processing component
		      \item Algorithm bias feature
		      \item Data protection component
	      \end{itemize}
	The major concern of this system is the security of user data provided by E-Risk. It must be known that data leaks from the system would be a potential hazard to these users. Data ingestion, processing, protection components of the system must be analyzed. 
\end{enumerate}

\section{Critical Assumptions}

\wss{These assumptions that are made about the software or system.  You should
minimize the number of assumptions that remove potential hazards.  For instance,
you could assume a part will never fail, but it is generally better to include
this potential failure mode.}

\section{Failure Mode and Effect Analysis}

\wss{Include your FMEA table here}

\section{Safety and Security Requirements}

\begin{enumerate}

    \item[SR1.] User data must not be shared or re-used in any system not part of this system
    \item[] \emph{Rationale:}  As users have an expectation that their personal data will be handled with care, sharing it with other systems will not guarantee their safety as it will not be under our control.
    \item[] \emph{Associated Hazards:} H1-1

    \item[SR2.] Sensitive user data is must not be present within the results generated
    \item[] \emph{Rationale:}  This is to ensure legal compliance and uphold ethical and professional standards. In addition, exposing people's PPI could lead to unauthorized access, data breaches, or privacy violations.
    \item[] \emph{Associated Hazards:} H1-1
\end{enumerate}


\section{Roadmap}

\wss{Which safety requirements will be implemented as part of the capstone timeline?
Which requirements will be implemented in the future?}

\end{document}