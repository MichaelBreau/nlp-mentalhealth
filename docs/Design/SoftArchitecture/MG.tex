\documentclass[12pt, titlepage]{article}

\usepackage{fullpage}
\usepackage[round]{natbib}
\usepackage{multirow}
\usepackage{booktabs}
\usepackage{tabularx}
\usepackage{graphicx}
\usepackage{float}
\usepackage{hyperref}
\hypersetup{
    colorlinks,
    citecolor=blue,
    filecolor=black,
    linkcolor=red,
    urlcolor=blue
}

%% Comments

\usepackage{color}

\newif\ifcomments\commentstrue %displays comments
%\newif\ifcomments\commentsfalse %so that comments do not display

\ifcomments
\newcommand{\authornote}[3]{\textcolor{#1}{[#3 ---#2]}}
\newcommand{\todo}[1]{\textcolor{red}{[TODO: #1]}}
\else
\newcommand{\authornote}[3]{}
\newcommand{\todo}[1]{}
\fi

\newcommand{\wss}[1]{\authornote{blue}{SS}{#1}} 
\newcommand{\plt}[1]{\authornote{magenta}{TPLT}{#1}} %For explanation of the template
\newcommand{\an}[1]{\authornote{cyan}{Author}{#1}}

%% Common Parts

\newcommand{\progname}{Natural Language Processing for Mental Health Risk Prediction} % PUT YOUR PROGRAM NAME HERE
\newcommand{\authname}{Team 13, The Cognitive Care Crew
\\ Jessica Dawson
\\ Michael Breau
\\ Matthew Curtis
\\ Benjamin Chinnery
\\ Yaruo Tian} % AUTHOR NAMES                  

\usepackage{hyperref}
    \hypersetup{colorlinks=true, linkcolor=blue, citecolor=blue, filecolor=blue,
                urlcolor=blue, unicode=false}
    \urlstyle{same}
                                


\newcounter{acnum}
\newcommand{\actheacnum}{AC\theacnum}
\newcommand{\acref}[1]{AC\ref{#1}}

\newcounter{ucnum}
\newcommand{\uctheucnum}{UC\theucnum}
\newcommand{\uref}[1]{UC\ref{#1}}

\newcounter{mnum}
\newcommand{\mthemnum}{M\themnum}
\newcommand{\mref}[1]{M\ref{#1}}

\newcounter{edmnum}
\newcommand{\edmthemnum}{EDM\theedmnum}
\newcommand{\edmref}[1]{EDM\ref{#1}}

\setcounter{secnumdepth}{5}

\begin{document}

\title{Module Guide for \progname{}} 
\author{\authname}
\date{\today}

\maketitle

\pagenumbering{roman}

\section{Revision History}

\begin{tabularx}{\textwidth}{p{3cm}p{2cm}X}
\toprule {\bf Date} & {\bf Version} & {\bf Notes}\\
\midrule
January 17, 2024 & 1.0 & Revision 0\\
\bottomrule
\end{tabularx}

\newpage

\section{Reference Material}

This section records information for easy reference.

\subsection{Abbreviations and Acronyms}

\renewcommand{\arraystretch}{1.2}
\begin{tabular}{l l} 
  \toprule		
  \textbf{symbol} & \textbf{description}\\
  \midrule 
  AC & Anticipated Change\\
  DAG & Directed Acyclic Graph \\
  M & Module \\
  MG & Module Guide \\
  OS & Operating System \\
  R & Requirement\\
  SC & Scientific Computing \\
  SRS & Software Requirements Specification\\
  UC & Unlikely Change \\
  \bottomrule
\end{tabular}\\

\newpage

\tableofcontents

\listoftables

\listoffigures

\newpage

\pagenumbering{arabic}

\section{Introduction}

Decomposing a system into modules is a commonly accepted approach to developing
software.  A module is a work assignment for a programmer or programming
team~\citep{ParnasEtAl1984}.  We advocate a decomposition
based on the principle of information hiding~\citep{Parnas1972a}.  This
principle supports design for change, because the ``secrets'' that each module
hides represent likely future changes.  Design for change is valuable in SC,
where modifications are frequent, especially during initial development as the
solution space is explored.  

Our design follows the rules layed out by \citet{ParnasEtAl1984}, as follows:
\begin{itemize}
\item System details that are likely to change independently should be the
  secrets of separate modules.
\item Each data structure is implemented in only one module.
\item Any other program that requires information stored in a module's data
  structures must obtain it by calling access programs belonging to that module.
\end{itemize}

After completing the first stage of the design, the Software Requirements
Specification (SRS), the Module Guide (MG) is developed~\citep{ParnasEtAl1984}. The MG
specifies the modular structure of the system and is intended to allow both
designers and maintainers to easily identify the parts of the software.  The
potential readers of this document are as follows:

\begin{itemize}
\item New project members: This document can be a guide for a new project member
  to easily understand the overall structure and quickly find the
  relevant modules they are searching for.
\item Maintainers: The hierarchical structure of the module guide improves the
  maintainers' understanding when they need to make changes to the system. It is
  important for a maintainer to update the relevant sections of the document
  after changes have been made.
\item Designers: Once the module guide has been written, it can be used to
  check for consistency, feasibility, and flexibility. Designers can verify the
  system in various ways, such as consistency among modules, feasibility of the
  decomposition, and flexibility of the design.
\end{itemize}

The rest of the document is organized as follows. Section
\ref{SecChange} lists the anticipated and unlikely changes of the software
requirements. Section \ref{SecMH} summarizes the module decomposition that
was constructed according to the likely changes. Section \ref{SecConnection}
specifies the connections between the software requirements and the
modules. Section \ref{SecMD} gives a detailed description of the
modules. Section \ref{SecTM} includes two traceability matrices. One checks
the completeness of the design against the requirements provided in the SRS. The
other shows the relation between anticipated changes and the modules. Section
\ref{SecUse} describes the use relation between modules.

\section{General Structure of an NLP Pipeline} \label{SecPipeline}

All NLP pipelines in this system can be thought of as following a general structure. This is not a hard description of the structure each diagnosis component will use but a soft description of what the system's NLP pipelines will look like. This pattern is likely to show up throughout the system in various locations and is included here to provide context and a conceptual model the reader can use for the rest of the document. The proposed structure is as follows: \newline

\noindent Input $\rightarrow$ Tokenizer $\rightarrow$ Representation Model $\rightarrow$ Prediction Model $\rightarrow$ Predictions \newline

\noindent Where the Prediciton Model can be further broken up into: \newline

\noindent $\rightarrow$ Transformation $\rightarrow$ ... $\rightarrow$ Transformation $\rightarrow$ \newline

\noindent A brief description of each stage is as follows:

\begin{itemize}
    \item Input: The input text to the pipeline.
    \item Tokenizer: Responsible for "cleaning" the text, can perform many different tasks: removing junk words (often referred to as stopwords) like the, and, a, etc.; expanding contractions; removing numerics and URLS. Leaves the text as text, just with the less useful features removed.
    \item Representation Model: Machine learning models can not directly operate on text, they need numerical inputs. The representation model converts from the text to some numerical format. Different representation models output different formats and encode different information into these formats.
    \item Prediction Model: Transforms the numerical format provided by the representation model into the mental health predictions of interest. The prediction model may transform the data mutliple times in different ways before reaching a final prediction.
    \begin{itemize}
        \item Transformation: A specific transformation in the prediction model's pipeline
    \end{itemize}
    \item Predictions: The final predictions produced by the pipeline.
\end{itemize}

Due to the interconnected nature of this pipeline and the research based nature of this project large changes to this pipeline are both expected and unavoidable. This is because changes in one section of pipeline can require large changes in other sections, trying a new prediction model may require an entirely different representation model do what it's meant tobe used.

\section{Anticipated and Unlikely Changes} \label{SecChange}

This section lists possible changes to the system. According to the likeliness
of the change, the possible changes are classified into two
categories. Anticipated changes are listed in Section \ref{SecAchange}, and
unlikely changes are listed in Section \ref{SecUchange}.

\subsection{Anticipated Changes} \label{SecAchange}

Anticipated changes are the source of the information that is to be hidden
inside the modules. Ideally, changing one of the anticipated changes will only
require changing the one module that hides the associated decision. The approach
adapted here is called design for
change.

\begin{description}
\item[\refstepcounter{acnum} \actheacnum \label{acInput}:] What format the input data takes.
\item[\refstepcounter{acnum} \actheacnum \label{acOutput}:] What format the output data takes.
\item[\refstepcounter{acnum} \actheacnum \label{acPipelineStages}:] What stages there are in a task's NLP pipeline.
\item[\refstepcounter{acnum} \actheacnum \label{acTokenization}:] How the raw text data is "cleaned" (see \ref{SecPipeline}: tokenizer).
\item[\refstepcounter{acnum} \actheacnum \label{acRepModel}:] How the cleaned text data is converted into numerical data.
\item[\refstepcounter{acnum} \actheacnum \label{acPredModel}:] How many manipulations of the numerical data are performed before producing a prediction.
\item[\refstepcounter{acnum} \actheacnum \label{acTrans}:] How the numerical data is manipulated to produce predictions.
\item[\refstepcounter{acnum} \actheacnum \label{acOps}:] How the system will integrate with the tools the operations team is developing.
\end{description}

\subsection{Unlikely Changes} \label{SecUchange}

The module design should be as general as possible. However, a general system is
more complex. Sometimes this complexity is not necessary. Fixing some design
decisions at the system architecture stage can simplify the software design. If
these decision should later need to be changed, then many parts of the design
will potentially need to be modified. Hence, it is not intended that these
decisions will be changed.

\begin{description}
\item[\refstepcounter{ucnum} \uctheucnum \label{ucIO}:] Input/Output devices
  (Input: File, Output: File).
\item[\refstepcounter{ucnum} \uctheucnum \label{ucTasks}:] What diagnosis tools the system will provide.
\end{description}

\section{Module Hierarchy} \label{SecMH}

This section provides an overview of the module design. Modules are summarized
in a hierarchy decomposed by secrets in Table \ref{TblMH}. The modules listed
below, which are leaves in the hierarchy tree, are the modules that will
actually be implemented. As the details of the NLP pipelines are not as of yet determined some modules are left quite general and vague (ie. how many transformation modules will be used, what representation models will be used, etc.).
\begin{description}
\item [\refstepcounter{mnum} \mthemnum \label{mHH}:] Hardware-Hiding Module
\item [\refstepcounter{mnum} \mthemnum \label{m2}:] Search for Symptoms of Depression Module
\item [\refstepcounter{mnum} \mthemnum \label{ANOM1}:] Anorexia Main Module
\item [\refstepcounter{mnum} \mthemnum \label{ANOM2}:] Anorexia Parser Module
\item [\refstepcounter{mnum} \mthemnum \label{ANOM3}:] Anorexia Training Module
\item [\refstepcounter{mnum} \mthemnum \label{ANOM4}:] Anorexia Predictor Module
\item [\refstepcounter{edmnum} \edmthemnum \label{edmI}:] ED Input
\item [\refstepcounter{edmnum} \edmthemnum \label{edmO}:] ED Output
\item [\refstepcounter{edmnum} \edmthemnum \label{edmPipeline}:] ED Pipeline Manager
\item [\refstepcounter{edmnum} \edmthemnum \label{edmToken}:] ED Tokenizer
\item [\refstepcounter{edmnum} \edmthemnum \label{edmRM}:] ED Representation Model
\item [\refstepcounter{edmnum} \edmthemnum \label{edmPM}:] ED Prediction Model
\item [\refstepcounter{edmnum} \edmthemnum \label{edmTrans}:] ED Transformation 1...n
\end{description}


\begin{table}[h!]
\centering
\begin{tabular}{p{0.2\textwidth} p{0.3\textwidth} p{0.4\textwidth}}
\toprule
\textbf{Level 1} & \textbf{Level 2} & \textbf{Level 3}\\
\midrule

{Hardware-Hiding} & None \\
\midrule

\multirow{3}{*}{Behaviour-Hiding} & \\ \cline{2-3}
& \multirow{2}{*}{Eating Disorder IO} & ED Input\\
& & ED Output\\
\midrule

\multirow{3}{*}{Software Decision} 
& Depression Module\\ \cline{2-3}
& \multirow{4}{*}{Anorexia Pipeline} & Anorexia Main\\ 
& & Anorexia Parser\\
& & Anorexia Training\\
& & Anorexia Predictor\\ \cline{2-3}
& \multirow{5}{*}{Eating Disorder Pipeline} & ED Pipeline Manager\\
& & ED Tokenizer\\
& & ED Representation Model\\
& & ED Prediction Model\\
& & ED Transformation 1...n (represents multiple modules)\\
\bottomrule

\end{tabular}
\caption{Module Hierarchy}
\label{TblMH}
\end{table}

\newpage

\section{Connection Between Requirements and Design} \label{SecConnection}

The design of the system is intended to satisfy the requirements developed in
the SRS. In this stage, the system is decomposed into modules. The connection
between requirements and modules is listed in Table~\ref{TblRT}.

\section{Module Decomposition} \label{SecMD}

Modules are decomposed according to the principle of ``information hiding''
proposed by \citet{ParnasEtAl1984}. The \emph{Secrets} field in a module
decomposition is a brief statement of the design decision hidden by the
module. The \emph{Services} field specifies \emph{what} the module will do
without documenting \emph{how} to do it. For each module, a suggestion for the
implementing software is given under the \emph{Implemented By} title. If the
entry is \emph{OS}, this means that the module is provided by the operating
system or by standard programming language libraries.  \emph{\progname{}} means the
module will be implemented by the \progname{} software.

Only the leaf modules in the hierarchy have to be implemented. If a dash
(\emph{--}) is shown, this means that the module is not a leaf and will not have
to be implemented.

\subsection{Hardware Hiding Modules (\mref{mHH})}

\begin{description}
\item[Secrets:]The data structure and algorithm used to implement the virtual
  hardware.
\item[Services:]Serves as a virtual hardware used by the rest of the
  system. This module provides the interface between the hardware and the
  software. So, the system can use it to display outputs or to accept inputs.
\item[Implemented By:] OS
\end{description}

\subsection{Behaviour-Hiding Module}

\begin{description}
\item[Secrets:]The contents of the required behaviours.
\item[Services:]Includes programs that provide externally visible behaviour of
  the system as specified in the software requirements specification (SRS)
  documents. This module serves as a communication layer between the
  hardware-hiding module and the software decision module. The programs in this
  module will need to change if there are changes in the SRS.
\item[Implemented By:] Depression IO, Anorexia IO, Eating Disorder IO
\end{description}

\subsubsection{Eating Disorder IO (\edmref{edmI}, \edmref{edmO})}
\begin{description}
\item[Secrets:]How input and output is handled for the eating disorder diagnosis tool.
\item[Services:]Includes programs that provide reading from an input file and outputting to an output file for the eating disorder diagnosis tool.
\item[Implemented By:] ed\_input.py, ed\_output.py
\end{description}

\paragraph{ED Input (\edmref{edmI})}
\begin{description}
\item[Secrets:]What format the input file is provided in.
\item[Services:] Provides functionality for reading user posts from a file and formatting them into a data structure usable by the rest of the system.
\item[Implemented By:] ed\_input.py
\item[Type of Module:] Library
\end{description}

\paragraph{ED Output (\edmref{edmO})}
\begin{description}
\item[Secrets:]What format the output file is written in.
\item[Services:] Provides functionality for writing predictions to a file.
\item[Implemented By:] ed\_output.py
\item[Type of Module:] Library
\end{description}

\subsection{Software Decision Module}
\subsubsection{Search for Symptoms of Depression Module (\mref{m2})}
\begin{description}
\item[Secrets:]The format and structures of detecting symptoms of depression data objects.
\item[Services:] Sorts and ranks 1000 sentences of users to a BDI Questionnaire with 21 questions in correspondence. Each question of the questionnaire will correspond to 1000 related sentences.
\item[Implemented By:] Natural Language Processing for Mental Health Risk Prevention
\end{description}

\subsubsection{Anorexia Main Module (\mref{ANOM1})}
\begin{description}
\item[Secrets:] The Natural Language processor which will combine the parser, training, and predictor modules to compute anorexia risk in new documents.
\item[Services:] Anorexia training data
  do not provide direct interaction with the user. 
\item[Implemented By:] task2\_main
\end{description}

\subsubsection{Anorexia Parser Module (\mref{ANOM2})}
\begin{description}
\item[Secrets:] The parser for taking user posts and converting them into a format readable by the nlp model.
\item[Services:] Anorexia training data
  do not provide direct interaction with the user. 
\item[Implemented By:] task2\_parser
\end{description}

\subsubsection{Anorexia Training Module (\mref{ANOM3})}
\begin{description}
\item[Secrets:] The Natural Language processor which will train the model on parsed data.
\item[Services:] Anorexia training data
  do not provide direct interaction with the user. 
\item[Implemented By:] task2\_training
\end{description}

\subsubsection{Anorexia Predictor Module (\mref{ANOM4})}
\begin{description}
\item[Secrets:] The Natural Language processor which will analyze inputted user data to determine a risk for anorexia
\item[Services:] Anorexia training data
  do not provide direct interaction with the user. 
\item[Implemented By:] task2\_predictor
\end{description}

\subsubsection{Eating Disorder Pipeline (-)}
\begin{description}
\item[Secrets:] The structure and design of the eating disorder NLP pipeline.
\item[Services:] Includes programs that manage the overall pipeline and others that define the specific transformations used to produce a prediction.
\item[Implemented By:] ED Pipeline Manager, ED Tokenizer, ED Representation Model, ED Prediction Model, ED Transformation 1...n
\end{description}

\paragraph{ED Pipeline Manager (\edmref{edmPipeline})}
\begin{description}
\item[Secrets:] What components the NLP pipeline for the eating disorder task is comprised of.
\item[Services:] Orchestrates and runs all the components of the eating disorder NLP pipeline.
\item[Implemented By:] ed\_pipeline.py
\item[Type of Module:] Library
\end{description}

\paragraph{ED Tokenizer (\edmref{edmToken})}
\begin{description}
\item[Secrets:] How the raw text data is cleaned in preparation of the model.
\item[Services:] Provides a routine for the cleaning and stripping of less useful information from raw text.
\item[Implemented By:] ed\_tokenizer.py
\item[Type of Module:] Library
\end{description}

\paragraph{ED Representation Model (\edmref{edmRM})}
\begin{description}
\item[Secrets:] How text data is converted into a numerical format.
\item[Services:] Provides a routine for converting text data into a numerical format.
\item[Implemented By:] ed\_rep\_model.py
\item[Type of Module:] Library
\end{description}

\paragraph{ED Prediction Model (\edmref{edmPM})}
\begin{description}
\item[Secrets:] The overall structure of the transformations used to produce a prediction from the numerical representation of the text.
\item[Services:] Orchestrates and runs all the transformations that make up the prediction model.
\item[Implemented By:] ed\_pred\_model.py
\item[Type of Module:] Library
\end{description}

\paragraph{ED Transformation 1...n (\edmref{edmTrans})}
\begin{description}
\item[Note:] Currently represents multiple modules, when a clearer picture of what structure the eating disorder pipeline will take on is reached this will be updated and expanded into multiple specific transformations
\item[Secrets:] What transformations are used to produce a prediction from the numerical representation of the text.
\item[Services:] Provides a routine for transforming numerical data from one form to another.
\item[Implemented By:] ed\_transformation1...n.py
\item[Type of Module:] Library
\end{description}

\section{Traceability Matrix} \label{SecTM}

This section shows two traceability matrices: between the modules and the
requirements and between the modules and the anticipated changes.

% the table should use mref, the requirements should be named, use something
% like fref
\begin{table}[H]
\centering
\begin{tabular}{p{0.2\textwidth} p{0.6\textwidth}}
\toprule
\textbf{Req.} & \textbf{Modules}\\
\midrule
T1FR-1 & \mref{m2}\\
T1FR-2 & \mref{m2}\\
T1FR-3 & \mref{m2}\\
T2FR-1 & \mref{ANOM1}, \mref{ANOM2}\\
T2FR-2 & \mref{ANOM1}, \mref{ANOM4}\\
T2FR-3 & \mref{ANOM1}, \mref{ANOM3}, \mref{ANOM4}\\
T2FR-4 & \mref{ANOM1}, \mref{ANOM4}\\
T3FR-1 & \edmref{edmI}\\
T3FR-2 & \edmref{edmO}\\
T3FR-3 & \edmref{edmO}\\
GR1 & \mref{m2}, \mref{ANOM1}, \edmref{edmI}, \edmref{edmO}\\
GR2 & \mref{m2}, \mref{ANOM4}, \edmref{edmO}\\
SR1 & \mref{m2}, \mref{ANOM1}, \edmref{edmO}\\
SR2 & \mref{m2}, \mref{ANOM4}, \edmref{edmO}\\

\bottomrule
\end{tabular}
\caption{Trace Between Requirements and Modules}
\label{TblRT}
\end{table}

\begin{table}[H]
\centering
\begin{tabular}{p{0.2\textwidth} p{0.6\textwidth}}
\toprule
\textbf{AC} & \textbf{Modules}\\
\midrule
\acref{acInput} & \mref{m2}, \mref{ANOM2}, \edmref{edmI}\\
\acref{acOutput} & \mref{m2}, \mref{ANOM4}, \edmref{edmO}\\
\acref{acPipelineStages} & \mref{m2}, \mref{ANOM1}, \edmref{edmPipeline}\\
\acref{acTokenization} & \mref{m2}, \mref{ANOM2}, \edmref{edmToken}\\
\acref{acRepModel} & \mref{m2}, \mref{ANOM3}, \edmref{edmRM}\\
\acref{acPredModel} & \mref{m2}, \mref{ANOM3}, \edmref{edmPM}\\
\acref{acTrans} & \mref{m2}, \mref{ANOM4}, \edmref{edmTrans}\\
\acref{acOps} & \mref{m2}, \mref{ANOM1}\\

\bottomrule
\end{tabular}
\caption{Trace Between Anticipated Changes and Modules}
\label{TblACT}
\end{table}

\section{Use Hierarchy Between Modules} \label{SecUse}

In this section, the uses hierarchy between modules is
provided. \citet{Parnas1978} said of two programs A and B that A {\em uses} B if
correct execution of B may be necessary for A to complete the task described in
its specification. That is, A {\em uses} B if there exist situations in which
the correct functioning of A depends upon the availability of a correct
implementation of B.  Figure \ref{FigUH} illustrates the use relation between
the modules. It can be seen that the graph is a directed acyclic graph
(DAG). Each level of the hierarchy offers a testable and usable subset of the
system, and modules in the higher level of the hierarchy are essentially simpler
because they use modules from the lower levels.

\begin{figure}[H]
\centering
%\includegraphics[width=0.7\textwidth]{UsesHierarchy.png}
\caption{Use hierarchy among modules}
\label{FigUH}
\end{figure}

\section{Timeline}

\begin{itemize}
    \item Module: \edmref{edmI} Finish Date: Jan 20 2024 Assignee(s): Jessica
    \item Module: \edmref{edmO} Finish Date: Jan 22 2024 Assignee(s): Jessica
    \item Module: \edmref{edmPipeline} Finish Date: Jan 24 2024 Assignee(s): Jessica
    \item Module: \edmref{edmToken} Finish Date: Jan 26 2024 Assignee(s): Jessica
    \item Module: \edmref{edmRM} Finish Date: Jan 28 2024 Assignee(s): Jessica
    \item Module: \edmref{edmPM} Finish Date: Jan 28 2024 Assignee(s): Jessica
    \item Module: \edmref{edmTrans} Finish Date: Feb 1 2024 Assignee(s): Jessica
    \item Module: \mref{m2} Finish Date: Feb 1 2024 Assignee(s): Franklin, Matthew
    \item Module: \mref{ANOM1} Finish Date: Feb 1 2024 Assignee(s): Michael, Ben
    \item Module: \mref{mHH} Finish Date: Feb 1 2024 Assignee(s): All
    \item Testing: All Modules Finish Date: Feb 4 2024 Assignee(s): All
\end{itemize}

%\section*{References}

\bibliographystyle {plainnat}
\bibliography{../../../refs/References}

\newpage{}

\end{document}